\documentclass[a4paper]{scrartcl}

% font/encoding packages
\usepackage[utf8]{inputenc}
\usepackage[T1]{fontenc}
\usepackage{lmodern}
\usepackage[ngerman]{babel}

% math
\usepackage{amsmath, amssymb, amsfonts, amsthm, mathtools}
\allowdisplaybreaks
\newtheorem*{behaupt}{Behauptung}
\usepackage{siunitx}

% tikz
\usepackage{tikz}
\usetikzlibrary{graphs}
\usetikzlibrary{arrows}
\usetikzlibrary{positioning}
\usepackage{pgfplots}
\usepgfplotslibrary{fillbetween}

% misc
\usepackage{enumerate}
\usepackage{algorithm}
\usepackage{algpseudocode}
\usepackage{listings}
\usepackage[section]{placeins}
                    
% macros
\newcommand{\linf}[1]{\lim_{#1 \to \infty}}
\newcommand{\gdw}{\Leftrightarrow}
\newcommand{\dif}{\mathop{}\!\mathrm{d}}
\newcommand{\bigo}{\mathcal{O}}

% last package
\usepackage{hyperref}


\title{Algorithmen und Datenstrukturen}
\subtitle{Übungsblatt 6 - Aufgabe 5}
\author{
    anonymous
}
\date{zum 13. Januar 2015}

\begin{document}
\maketitle

\begin{enumerate}[(a)]
    \item
        \begin{behaupt}
            Falls für einen Schnitt $C = (S, V \backslash S)$ von $G$ eine
            eindeutig bestimmte Schnittkante $e$ mit minimalem Gewicht
            existiert, so ist $e$ in jedem MST vorhanden.
        \end{behaupt}
        \begin{proof}[Beweis durch Widerspruch]
            Angenommen es gäbe einen MST $T$ ohne $e$.
            Da MSTs zusammenhängend sind, würde durch Hinzunahme von $e$ ein
            Kreis entstehen.
            Also muss $T$ eine andere Schnittkante $f$ enthalten.
            Da $w(e) < w(f)$ gilt, kann durch Austauschen von $e$ und $f$ ein
            Spanning Tree $T'$ mit $w(T') < w(T)$ entstehen.
            Dies ist ein Widerspruch zu der Annahme, dass $T$ ein MST sei.
        \end{proof}
        \begin{behaupt}
            Falls für jeden Schnitt $(S, V \backslash S)$ von $G$ eine eindeutig
            bestimmte Schnittkante mit minimalem Gewicht existiert, so hat $G$
            einen eindeutig bestimmten minimalen Spannbaum.
        \end{behaupt}
        \begin{proof}
            Sei $CE$ die Menge der eindeutigen Schnittkanten mit minimalem
            Gewicht für alle Schnitte.
            Da die erste Behauptung gilt, müssen alle Kanten in $CE$ auch in
            einem MST vorkommen.
            Dieser MST kann wie folgt konstruiert werden:
            Betrachtet werden Schnitte $C_i = (A_i, V \backslash A_i)$, wobei
            initial $|A_0| = 1$ gilt und anschließend in jedem der $|V| - 1$
            Schritte ein weiterer Knoten zu $A_i$ hinzugefügt wird, welcher
            durch die Schnittkante mit minimalem Gewicht bestimmt wird.
            Diese ist für jeden $C_i$ vorhanden, eindeutig und muss in jedem
            MST vorhanden sein.
            Die Menge dieser Kanten ist $E'$ für den eindeutigen MST
            $T = (V, E')$.
        \end{proof}

    \item
        \begin{behaupt}
            Folgende Aussage gilt im Allgemeinen nicht: \\
            Für jeden Schnitt $(S, V \backslash S)$ von $G$ existiert eine
            eindeutig bestimmte Schnittkante mit minimalem Gewicht, falls $G$
            einen eindeutig bestimmten minimalen Spannbaum hat.
        \end{behaupt}
        \begin{proof}
            Sei $G = (V, E)$ ein vollständiger Graph mit
            \begin{equation}
                V = \left\{ a, b, c \right\}
            \end{equation}
            und
            \begin{equation}
                \begin{split}
                    w(a, b) &= 1 \\
                    w(a, c) &= 1 \\
                    w(b, c) &= 2
                \end{split}
            \end{equation}
            Offensichtlich ist der Baum
            \begin{equation}
                T = \left(V, \left\{ \{a,b\}, \{a,c\} \right\} \right)
            \end{equation}
            der einzige MST von $G$.
            Für den Schnitt $(\{a\}, \{b,c\})$ existiert jedoch keine eindeutig
            bestimmte Schnittkante mit minimalem Gewicht:
            Die beiden Schnittkanten $\{ a,b \}$ und $\{ a,c \}$ haben beide ein
            Kantengewicht von $1$.
        \end{proof}

\end{enumerate}

\end{document}
