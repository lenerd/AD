\documentclass[a4paper]{scrartcl}

% font/encoding packages
\usepackage[utf8]{inputenc}
\usepackage[T1]{fontenc}
\usepackage{lmodern}
\usepackage[ngerman]{babel}

% math
\usepackage{amsmath, amssymb, amsfonts, amsthm, mathtools}
\allowdisplaybreaks
\newtheorem*{behaupt}{Behauptung}
\usepackage{siunitx}

% misc
\usepackage{enumerate}
\usepackage{algorithm}
\usepackage{algpseudocode}
\usepackage[section]{placeins}

% tikz
\usepackage{tikz}
\usetikzlibrary{graphs}
\usetikzlibrary{arrows}
\usetikzlibrary{positioning}
\usepackage{pgfplots}
\usepgfplotslibrary{fillbetween}
                    
% macros
\newcommand{\linf}[1]{\lim_{#1 \to \infty}}
\newcommand{\gdw}{\Leftrightarrow}
\newcommand{\dif}{\mathop{}\!\mathrm{d}}
\newcommand{\bigo}{\mathcal{O}}


\title{Algorithmen und Datenstrukturen}
\subtitle{Übungsblatt 6 - Aufgabe 1}
\author{
    anonymous
}
\date{zum 13. Januar 2015}

\begin{document}
\maketitle

\begin{algorithm}
    \caption{\textsc{Modified Floyd-Warshall}}
    \label{alg:fw}
    \begin{algorithmic}[1]
        \Procedure{FloydWarshall'}{$W$}
            \State $n \gets$ \text{number of vertices}
            \State $D^{(0)} \gets W$
            \Comment Matrix of size $n \times n$
            \State Let $\Pi$ be a new $n \times n$ Matrix
            \Comment predecessor matrix
            \State $\pi(u, v) \gets
            \begin{cases}
                NULL & \text{if } u = v \text{ or } w(i,j) = \infty \\
                u & \text{else} \\
            \end{cases}$
            \For {$k = 1, \ldots, n$}
                \State Let $D^{(k)}$ be a new $n \times n$ Matrix
                \For {$s = 1, \ldots, n$}
                    \For {$t = 1, \ldots, n$}
                        \State $d_k(s,t) \gets \min \left\{ d_{k-1}(s,t),
                        d_{k-1}(s,k) + d_{k-1}(k,t) \right\}$
                        \If {$d_k(s,t) \neq d_{k-1}(s,t)$}
                            \State $\pi(s,t) \gets \pi(k, t)$
                        \EndIf
                    \EndFor
                \EndFor
            \EndFor
        \EndProcedure
        \Procedure{PrintPath}{$\Pi, s, t$}
            \If {$s =  t$}
                \State print $s$
            \ElsIf {$\pi(s,t) = -1$}
                \State print „there is no path from $s$ to $t$“
            \Else
                \State \Call{PrintPath}{$\Pi, s, \pi(s, t)$}
                \State print $t$
            \EndIf
        \EndProcedure
    \end{algorithmic}
\end{algorithm}

\section*{Korrektheit}
\begin{behaupt}
    Die Matrix ist die Vorgängermatrix für die kürzesten Pfade zwischen allen
    Knoten.
    $\pi(u,v) = x$ ist der vorletzte Knoten auf dem Pfad von $u$ nach $v$.
    Der Pfad ist also $u, \ldots, x, v$.
\end{behaupt}
\begin{proof} \hfill \\
    In jeder Iteration der Schleife enthält $\pi(u,v)$ den Vorgänger von $v$ auf
    dem zu der Zeit kürzesten bekannten Pfad von $u$ nach $v$.

    \subparagraph{1. Induktionsanfang:} \hfill \\
    Während der Initialisierung sind nur Pfade bekannt, welche aus einer Kante
    im Graphen bestehen.
    Existiert keine Kante zwischen $s$ und $t$, so gilt $w(s,t) = \infty$ und
    $\pi(s,t)$ wird mit $NULL$ initialisiert.
    Gilt $s = t$, so besteht der Pfad nur aus $s$ und ein Vorgänger ist nicht
    vorhanden; auch hier wird mit $NULL$ initialisiert.
    Ist eine Kante von $s$ nach $t$ vorhanden, so wird der $\pi(s,t) = s$
    gesetzt, da der kürzeste bekannte Pfad $s, t$ ist.
    \\
    \\
    Angenommen die Behauptung gelte in einer Iteration $k-1$ ($k$ fest).

    \subparagraph{2. Induktionsschritt:} \hfill \\
    Ändert sich der kürzeste bekannte Pfad in der $k$-ten Iteration
    (in Zeile~11 geprüft), so muss ein neuer Vorgänger gesetzt werden.
    In diesem Fall gilt $d_k(s,t) = d_{k-1}(s,k) + d_{k-1}(k,t)$.
    Der neue, kürzere Pfad besteht also aus dem Pfad von $s$ nach $k$
    konkateniert mit dem Pfad von $k$ nach $t$.
    Der neue Vorgänger $\pi(s,t)$ ist also der Vorgänger von $t$ auf dem Pfad
    von $k$ nach $t$: $\pi(k,t)$ (Zeile~12).
    \\
    \\
    Da davon ausgegangen wird, dass der Floyd-Warshall-Algorithmus korrekt ist
    und nach der letzten Iteration der äußeren Schleife die Länge des kürzesten
    Pfades liefert, ist nach 1. und 2. der richtige Vorgänger auf diesem Pfad
    in $\Pi$ eingetragen.

    Der Pfad lässt sich einfach mit der rekursiven Funktion \textsc{PrintPath}
    ausgeben.


\end{proof}

\end{document}
