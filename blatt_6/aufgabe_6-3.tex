\documentclass[a4paper]{scrartcl}

% font/encoding packages
\usepackage[utf8]{inputenc}
\usepackage[T1]{fontenc}
\usepackage{lmodern}
\usepackage[ngerman]{babel}

% math
\usepackage{amsmath, amssymb, amsfonts, amsthm, mathtools}
\allowdisplaybreaks
\newtheorem*{behaupt}{Behauptung}
\usepackage{siunitx}

% misc
\usepackage{enumerate}
\usepackage{algorithm}
\usepackage{algpseudocode}
\usepackage[section]{placeins}

% tikz
\usepackage{tikz}
\usetikzlibrary{graphs}
\usetikzlibrary{arrows}
\usetikzlibrary{positioning}
\usepackage{pgfplots}
\usepgfplotslibrary{fillbetween}
                    
% macros
\newcommand{\linf}[1]{\lim_{#1 \to \infty}}
\newcommand{\gdw}{\Leftrightarrow}
\newcommand{\dif}{\mathop{}\!\mathrm{d}}
\newcommand{\bigo}{\mathcal{O}}


\title{Algorithmen und Datenstrukturen}
\subtitle{Übungsblatt 6 - Aufgabe 3}
\author{
    anonymous
}
\date{zum 13. Januar 2015}

\begin{document}
\maketitle

Ausführung von Kruskals Algorithmus
\begin{enumerate}
    \item Iteration \\
        Kanten
        \begin{equation}
            A = \left\{
                \{ 3,4 \}
            \right\}
        \end{equation}
        
        Partition
        \begin{equation}
            P = \left\{
                \{ 1 \},
                \{ 2 \},
                \{ 3,4 \},
                \{ 5 \},
                \{ 6 \},
                \{ 7 \}
            \right\}
        \end{equation}
        
    \item Iteration \\
        Kanten
        \begin{equation}
            A = \left\{
                \{ 3,4 \},
                \{ 3,5 \}
            \right\}
        \end{equation}
        
        Partition
        \begin{equation}
            P = \left\{
                \{ 1 \},
                \{ 2 \},
                \{ 3, 4, 5 \},
                \{ 6 \},
                \{ 7 \}
            \right\}
        \end{equation}
        
    \item Iteration \\
        Kanten
        \begin{equation}
            A = \left\{
                \{ 3,4 \},
                \{ 3,5 \},
                \{ 2,7 \}
            \right\}
        \end{equation}
        
        Partition
        \begin{equation}
            P = \left\{
                \{ 1 \},
                \{ 2, 7 \},
                \{ 3, 4, 5 \},
                \{ 6 \}
            \right\}
        \end{equation}
        
    \item Iteration:\\
        Kanten
        \begin{equation}
            A = \left\{
                \{ 3,4 \},
                \{ 3,5 \},
                \{ 2,7 \},
                \{ 5,7 \}
            \right\}
        \end{equation}
        
        Partition
        \begin{equation}
            P = \left\{
                \{ 1 \},
                \{ 2, 3, 4, 5, 7 \},
                \{ 6 \}
            \right\}
        \end{equation}
        
    \item Iteration:\\
        Kanten
        \begin{equation}
            A = \left\{
                \{ 3,4 \},
                \{ 3,5 \},
                \{ 2,7 \},
                \{ 5,7 \},
                \{ 6,7 \}
            \right\}
        \end{equation}
        
        Partition
        \begin{equation}
            P = \left\{
                \{ 1 \},
                \{ 2, 3, 4, 5, 6, 7 \}
            \right\}
        \end{equation}
        
    \item Iteration:\\
        Kanten
        \begin{equation}
            A = \left\{
                \{ 3, 4 \},
                \{ 3, 5 \},
                \{ 2, 7 \},
                \{ 5, 7 \},
                \{ 6, 7 \},
                \{ 1, 6 \}
            \right\}
        \end{equation}
        
        Partition
        \begin{equation}
            P = \left\{
                \{ 1, 2, 3, 4, 5, 6, 7 \}
            \right\}
        \end{equation}
        

\end{enumerate}
\begin{figure}[h]
    \centering
    \begin{tikzpicture}[
            auto,
            scale=2,
        ]
        \tikzstyle{vertex}=[
            draw,
            circle,
        ]
        \node [vertex] (v1) at (0,1) {$1$};
        \node [vertex] (v2) at (1,2) {$2$};
        \node [vertex] (v3) at (3,2) {$3$};
        \node [vertex] (v4) at (4,1) {$4$};
        \node [vertex] (v5) at (3,0) {$5$};
        \node [vertex] (v6) at (1,0) {$6$};
        \node [vertex] (v7) at (2,1) {$7$};

        % \draw (v1) to node {$10$} (v2);
        \draw (v1) to node {$8$}  (v6);
        % \draw (v2) to node {$10$} (v3);
        % \draw (v2) to node {$6$}  (v6);
        \draw (v2) to node {$3$}  (v7);
        \draw (v3) to node {$1$}  (v4);
        \draw (v3) to node {$2$}  (v5);
        % \draw (v3) to node {$4$}  (v7);
        % \draw (v4) to node {$5$}  (v5);
        % \draw (v5) to node {$9$}  (v6);
        \draw (v5) to node {$3$}  (v7);
        \draw (v6) to node {$4$}  (v7);
    \end{tikzpicture}
    \caption{MST des gegebenen Graphens}
\end{figure}

\end{document}
