\documentclass[a4paper]{scrartcl}

% font/encoding packages
\usepackage[utf8]{inputenc}
\usepackage[T1]{fontenc}
\usepackage{lmodern}
\usepackage[ngerman]{babel}

% math
\usepackage{amsmath, amssymb, amsfonts, amsthm, mathtools}
\allowdisplaybreaks
\newtheorem*{behaupt}{Behauptung}
\usepackage{siunitx}

% tikz
\usepackage{tikz}
\usetikzlibrary{graphs}
\usetikzlibrary{arrows}
\usetikzlibrary{positioning}
\usepackage{pgfplots}
\usepgfplotslibrary{fillbetween}

% misc
\usepackage{enumerate}
\usepackage{algorithm}
\usepackage{algpseudocode}
\usepackage{listings}
\usepackage[section]{placeins}
                    
% macros
\newcommand{\linf}[1]{\lim_{#1 \to \infty}}
\newcommand{\gdw}{\Leftrightarrow}
\newcommand{\dif}{\mathop{}\!\mathrm{d}}
\newcommand{\bigo}{\mathcal{O}}

% last package
\usepackage{hyperref}


\title{Algorithmen und Datenstrukturen}
\subtitle{Übungsblatt 6 - Aufgabe 2}
\author{
    anonymous
}
\date{zum 13. Januar 2015}

\begin{document}
\maketitle

\begin{enumerate}
    \item Iteration
        \begin{equation*}
            u = Frankfurt
        \end{equation*}
        \begin{align*}
            Mannheim.dist &= 85 \\
            Mannheim.\pi &= Frankfurt \\
            Würzburg.dist &= 217 \\
            Würzburg.\pi &= Frankfurt \\
            Kassel.dist &= 173 \\
            Kasse.\pi &= Frankfurt
        \end{align*}
        \begin{align*}
            Augsburg.status &= unreached \\
            Erfurt.status &= unreached \\
            Frankfurt.status &= settled \\
            Karlsruhe.status &= unreached \\
            Kassel.status &= labelchanged \\
            Mannheim.status &= labelchanged \\
            München.status &= unreached \\
            Nürnberg.status &= unreached \\
            Stuttgart.status &= unreached \\
            Würzburg.status &= labelchanged \\
        \end{align*}
        
    \item Iteration
        \begin{equation*}
            u = Würzburg
        \end{equation*}
        \begin{align*}
            Nürnberg.dist &= 320 \\
            Nürnberg.\pi &= Würzburg \\
            Erfurt.dist &= 403 \\
            Erfurt.\pi &= Würzburg
        \end{align*}
        \begin{align*}
            Augsburg.status &= unreached \\
            Erfurt.status &= labelchanged \\
            Frankfurt.status &= settled \\
            Karlsruhe.status &= unreached \\
            Kassel.status &= labelchanged \\
            Mannheim.status &= labelchanged \\
            München.status &= unreached \\
            Nürnberg.status &= labelchanged \\
            Stuttgart.status &= unreached \\
            Würzburg.status &= settled \\
        \end{align*}
        
    \item Iteration
        \begin{equation*}
            u = Mannheim
        \end{equation*}
        \begin{align*}
            Karlsruhe.dist &= 165 \\
            Karlsruhe.\pi &= Mannheim
        \end{align*}
        \begin{align*}
            Augsburg.status &= unreached \\
            Erfurt.status &= labelchanged \\
            Frankfurt.status &= settled \\
            Karlsruhe.status &= labelchanged \\
            Kassel.status &= labelchanged \\
            Mannheim.status &= settled \\
            München.status &= unreached \\
            Nürnberg.status &= labelchanged \\
            Stuttgart.status &= unreached \\
            Würzburg.status &= settled \\
        \end{align*}
        
    \item Iteration
        \begin{equation*}
            u = Nürnberg
        \end{equation*}
        \begin{align*}
            München.dist &= 487 \\
            München.\pi &= Nürnberg \\
            Stuttgart.dist &= 503 \\
            Stuttgart.\pi &= Nürnberg
        \end{align*}
        \begin{align*}
            Augsburg.status &= unreached \\
            Erfurt.status &= labelchanged \\
            Frankfurt.status &= settled \\
            Karlsruhe.status &= labelchanged \\
            Kassel.status &= labelchanged \\
            Mannheim.status &= settled \\
            München.status &= labelchanged \\
            Nürnberg.status &= settled \\
            Stuttgart.status &= labelchanged \\
            Würzburg.status &= settled \\
        \end{align*}
        
    \item Iteration
        \begin{equation*}
            u = München
        \end{equation*}
        Termination

\end{enumerate}

Der kürzeste Pfad von Frankfurt nach Münschen verläuft über
Würzburg und Nürnberg; die Länge beträgt \SI{487}{\kilo\meter}.

\end{document}
