\documentclass[a4paper]{scrartcl}

% font/encoding packages
\usepackage[utf8]{inputenc}
\usepackage[T1]{fontenc}
\usepackage{lmodern}
\usepackage[ngerman]{babel}

% math
\usepackage{amsmath, amssymb, amsfonts, amsthm, mathtools}
\allowdisplaybreaks
\newtheorem*{behaupt}{Behauptung}
\usepackage{siunitx}

% tikz
\usepackage{tikz}
\usetikzlibrary{graphs}
\usetikzlibrary{arrows}
\usetikzlibrary{positioning}
\usepackage{pgfplots}
\usepgfplotslibrary{fillbetween}

% misc
\usepackage{enumerate}
\usepackage{algorithm}
\usepackage{algpseudocode}
\usepackage{listings}
\usepackage[section]{placeins}
                    
% macros
\newcommand{\linf}[1]{\lim_{#1 \to \infty}}
\newcommand{\gdw}{\Leftrightarrow}
\newcommand{\dif}{\mathop{}\!\mathrm{d}}
\newcommand{\bigo}{\mathcal{O}}

% last package
\usepackage{hyperref}


\title{Algorithmen und Datenstrukturen}
\subtitle{Übungsblatt 6 - Aufgabe 4}
\author{
    anonymous
}
\date{zum 13. Januar 2015}

\begin{document}
\maketitle

Gegeben sei ein zusammenhängender kantengewichteter Graph $G = (V, E)$ und ein
minimaler Spannbaum $T = (V, E')$ von $G$.

\begin{enumerate}[(a)]
    \item
        \begin{behaupt}
            Es wird das Gewicht einer Kante $e \in E \backslash E'$ erhöht,
            $T$ ist auch ein minimaler Spannbaum des veränderten Graphen $G'$.
        \end{behaupt}
        \begin{proof}
            Angenommen $T$ sei kein MST in $G'$.
            Dann gibt es einen MST $T'_{G'} \neq T$ in $G'$.
            Nach diesen Annahmen muss $w(T'_{G'}) < w(T)$ gelten.
            \begin{enumerate}
                \item $T'_{G'}$ enthält $e$ \\
                    Da $w(e)$ erhöht wurde, muss für ein $T' = T'_{G'}$ in $G$ gelten
                    $w(T') < w(T'_{G'}) < w(T)$.

                \item $T'_{G'}$ enthält $e$ nicht \\
                    Dann gibt es ein $T' = T'_{G'}$ in $G$ geben für das gilt
                    $w(T') = w(T'_{G'}) < w(T)$.
            \end{enumerate}
            Dies sind Widersprüche zu der gegebenen Aussage, dass $T$ ein MST
            in $G$ ist.
            Daher muss obige Annahme falsch sein und $T$ ist auch ein minimaler
            Spannbaum des veränderten Graphen $G'$.
        \end{proof}

    \item

\end{enumerate}


\end{document}
