\documentclass[a4paper]{scrartcl}

% font/encoding packages
\usepackage[utf8]{inputenc}
\usepackage[T1]{fontenc}
\usepackage{lmodern}
\usepackage[ngerman]{babel}

% math
\usepackage{amsmath, amssymb, amsfonts, amsthm, mathtools}
\allowdisplaybreaks
\newtheorem*{behaupt}{Behauptung}
\usepackage{siunitx}

% misc
\usepackage{enumerate}
\usepackage{algorithm}
\usepackage{algpseudocode}
\usepackage[section]{placeins}

% tikz
\usepackage{tikz}
\usetikzlibrary{graphs}
\usetikzlibrary{arrows}
\usetikzlibrary{positioning}
\usepackage{pgfplots}
\usepgfplotslibrary{fillbetween}
                    
% macros
\newcommand{\linf}[1]{\lim_{#1 \to \infty}}
\newcommand{\gdw}{\Leftrightarrow}
\newcommand{\dif}{\mathop{}\!\mathrm{d}}
\newcommand{\bigo}{\mathcal{O}}


\title{Algorithmen und Datenstrukturen}
\subtitle{Übungsblatt 6 - Aufgabe 4}
\author{
    anonymous
}
\date{zum 13. Januar 2015}

\begin{document}
\maketitle

Gegeben sei ein zusammenhängender kantengewichteter Graph $G = (V, E)$ und ein
minimaler Spannbaum $T = (V, E')$ von $G$.

\begin{enumerate}[(a)]
    \item
        \begin{behaupt}
            Es wird das Gewicht einer Kante $e \in E \backslash E'$ erhöht,
            $T$ ist auch ein minimaler Spannbaum des veränderten Graphen $G'$.
        \end{behaupt}
        \begin{proof}
            Angenommen $T$ sei kein MST in $G'$.
            Dann gibt es einen MST $T'_{G'} \neq T$ in $G'$.
            Nach diesen Annahmen muss $w(T'_{G'}) < w(T)$ gelten.
            \begin{enumerate}
                \item $T'_{G'}$ enthält $e$ \\
                    Da $w(e)$ erhöht wurde, muss für ein $T' = T'_{G'}$ in $G$ gelten
                    $w(T') < w(T'_{G'}) < w(T)$.

                \item $T'_{G'}$ enthält $e$ nicht \\
                    Dann gibt es ein $T' = T'_{G'}$ in $G$ geben für das gilt
                    $w(T') = w(T'_{G'}) < w(T)$.
            \end{enumerate}
            Dies sind Widersprüche zu der gegebenen Aussage, dass $T$ ein MST
            in $G$ ist.
            Daher muss obige Annahme falsch sein und $T$ ist auch ein minimaler
            Spannbaum des veränderten Graphen $G'$.
        \end{proof}

    \item
        \begin{enumerate}[1.]
            \item $e \in E'$
                \begin{behaupt}
                    $T$ ist auch im veränderten Graphen $G'$ ein minimaler
                    Spannbaum.
                \end{behaupt}
                \begin{proof}
                    Sei $k > 0$ die Differenz, um die das Gewicht von $e$
                    verringert wird.
                    Da $e$ in $T$ enthalten ist, gilt $w'(T) = w(T) - k < w(T)$.
                    Angenommen $T$ sei kein MST in $G'$.
                    Sei $T' \neq T$ ein MST in $G'$.
                    \begin{enumerate}
                        \item $T'$ enthält $e$ \\
                            Es gilt $w(T') = w'(T') + k$
                            Da $T$ ein MST in $G$ ist, muss gelten
                            \begin{equation}
                                \begin{split}
                                    w(T) &\leq w(T') \\
                                    w(T) - k &\leq w(T') - k \\
                                    w'(T) &\leq w'(T')
                                \end{split}
                            \end{equation}
                            Da $T'$ ein MST in $G'$ ist, folgt aus obiger
                            Gleichung, dass auch $T$ ein MST in $G'$ sein muss.
                            
                        \item $T'$ enthält $e$ nicht \\
                            Es gilt $w(T') = w'(T')$
                            Da $T$ ein MST in $G$ ist, muss gelten
                            \begin{equation}
                                \begin{split}
                                    w(T) &\leq w(T') \\
                                    w'(T) < w(t) &\leq w(T') = w'(T')
                                \end{split}
                            \end{equation}
                            $T'$ kann also kein MST in $G'$ sein.
                    \end{enumerate}
                    Dies ist ein Widerspruch; also muss unsere Annahme, $T$ sei
                    kein MST von $G'$, falsch sein.

                \end{proof}

            \item $e \notin E'$ \\
                \begin{behaupt}
                    Mit folgendem Verfahren lässt sich mit Hilfe von $T$ ein
                    minimaler Spannbaum für den Graphen $G'$ konstruieren. \\

                    Sei
                    \begin{equation}
                        A = \left\{
                            f \in E' \ |\  w'(f) \leq w'(e)
                        \right\} \text{.}
                    \end{equation}
                    Ausgehend von $A$ wird Kruskals Algorithmus ausgeführt,
                    wobei alle Kanten $g$ mit $w'(g) < w'(e)$ übersprungen
                    werden und als Parition die Menge der
                    Zusammenhangskomponenten verwendet wird.
                \end{behaupt}
                \begin{proof}
                    Da die Kanten sortiert nach Gewicht in aufsteigender
                    Reihenfolge durchlaufen werden, ist der Verlauf zu Beginn
                    von Kruskals Algorithmus in $G$ und $G'$ der gleiche.
                    Alle Kanten mit Gewichten $< w'(e)$ sind entweder in beiden
                    MSTs vorhanden oder in keinem von beiden.
                    Anschließend wird der Kruskals Algorithmus wie üblich (mit
                    dem veränderten Gewicht) fortgesetzt.
                \end{proof}

        \end{enumerate}

\end{enumerate}


\end{document}
