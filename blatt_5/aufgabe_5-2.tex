\documentclass[a4paper]{scrartcl}

% font/encoding packages
\usepackage[utf8]{inputenc}
\usepackage[T1]{fontenc}
\usepackage{lmodern}
\usepackage[ngerman]{babel}

% math
\usepackage{amsmath, amssymb, amsfonts, amsthm, mathtools}
\allowdisplaybreaks
\newtheorem*{behaupt}{Behauptung}
\usepackage{siunitx}

% tikz
\usepackage{tikz}
\usetikzlibrary{graphs}
\usetikzlibrary{arrows}
\usetikzlibrary{positioning}
\usepackage{pgfplots}
\usepgfplotslibrary{fillbetween}

% misc
\usepackage{enumerate}
\usepackage{algorithm}
\usepackage{algpseudocode}
\usepackage{listings}
\usepackage[section]{placeins}
                    
% macros
\newcommand{\linf}[1]{\lim_{#1 \to \infty}}
\newcommand{\gdw}{\Leftrightarrow}
\newcommand{\dif}{\mathop{}\!\mathrm{d}}
\newcommand{\bigo}{\mathcal{O}}

% last package
\usepackage{hyperref}


\title{Algorithmen und Datenstrukturen}
\subtitle{Übungsblatt 5 - Aufgabe 2}
\author{
    anonymous
}
\date{zum 16. Dezember 2014}

\begin{document}
\maketitle

\begin{enumerate}[(a)]
    \item
        Das Problem, die Länge des kürzesten einfachen Pfades zu finden, ist nur
        dann definiert, wenn der betrachtete Graph $G$ stark zusammenhängend
        ist.
        Andernfalls existiert im allgemeinen Fall kein Pfad von $u$ nach $v$ für
        beliebige $u, v \in G.V$.
        Im Folgenden nehmen wir an, dass es sich um stark zusammenhängende
        Graphen handelt.

    \item
        Da uns \textsc{BellmanFord} nur $-\infty$ zurückgibt, wenn negative
        Zyklen genommen werden können, müssen wir uns die Pfadlängen auf andere
        Art besorgen.
        Da es einen kürzesten einfachen Pfad gibt, gibt es eine Teilmenge von
        $E$, welche die Kanten des Pfades enthält, jedoch nicht den negativen
        Zyklus, der das Ergebnis verfälscht hat.
        Die Idee ist, für jede Teilmenge $E'$ von $E$ einen neuen Graphen
        $G = (V, E')$ zu bauen und \textsc{BellmanFord} auf diesen auszuführen.
        Nach jedem Durchlauf schauen wir, ob für die Pfadlänge zu einem Knoten
        $v$ größer als $-\infty$ und kleiner als die bisher kürzeste Pfadlänge
        $v.dist^*$ ist.
        Damit ignorieren wir alle durch negative Zyklen verfälschten Werte und
        finden die Länge des kürzesten einfachen Pfads.
        Diese steht nach Termination in $v.dist^*$ für die Pfad von $s$ bis $v$.
        Das Vorgehen ist auch in Algorithmus \ref{alg:s4p} auf Seite
        \pageref{alg:s4p} beschrieben.

        \begin{algorithm}
            \caption{\textsc{Single-Source-Shortest-Simple-Path}}
            \label{alg:s4p}
            \begin{algorithmic}[1]
                \Procedure{S4P}{$G, s$}
                    \For {$v \in G.V$}
                        \State {$v.dist^* \gets \infty$}
                    \EndFor
                    \State $s.dist^* \gets 0$
                    \For {$E' \in \mathcal{P}(G.E)$}
                        \State $G' \gets (G.V, E')$
                        \State \Call{BellmanFord}{$G', s$}
                        \For {$v \in G.V$}
                            \If {$-\infty < v.dist < v.dist^*$}
                                \State $v.dist \gets v.dist$
                            \EndIf
                        \EndFor
                    \EndFor
                \EndProcedure
            \end{algorithmic}
        \end{algorithm}

        \paragraph{Korrektheit}
        Sei $\pi = s \ldots v$ der kürzeste einfache Pfad von $s$ nach $v$.
        Da der Algorithmus den Graphen mit jeder Teilmenge an Kanten durchgeht,
        treffen wir nach Konstruktion auf die Teilmenge
        \begin{equation}
            E' = \left\{ (i, j) \ \vert\  ij \text{ ist ein Teilwort von } \pi \right\} \text{ .}
        \end{equation}
        In dem Graphen $G' = (V, E')$ kann \textsc{BellmanFord} nicht durch
        negative Zyklen abgelenkt werden und gibt die korrekte Länge des Pfades
        $\pi$ zurück.
        Nach der Iteration der zweiten Schleife mit obigem $E'$ wird $v.dist^*$
        auf den richtigen (und damit kleinsten) Wert gesetzt.
        

    \item
        Die Laufzeit von Algorithmus \ref{alg:s4p} liegt für kein festes
        $k \in \mathbb{N}$ in $\mathcal{O} (n^k)$.

        Die erste und die dritte Schleife (Zeile 2) werden jeweils $|V|$-mal
        durchlaufen.
        Die zweite Schleife iteriert über alle Teilmengen von $G.E$, wird also
        $2^{|E|}$-mal ausgeführt.
        \textsc{BellmanFord} liegt in $\mathcal{O}(|V| \cdot |E|)$.
        Die Konstruktion der Potenzmenge braucht ebenfalls Zeit.
        
        Die Zeitkomplexität liegt also in
        \begin{equation}
            \Omega \left( 2^{|E|} \cdot |V| \cdot |E| \right) \text{ .}
        \end{equation}

        Bei Eingabe eines vollständigen Graphen wird die worst-case Laufzeit
        erreicht.

        Das bekannte Problem \emph{longest path problem}
        \footnote{\url{http://sarielhp.org/misc/funny/longestpath.mp3}},
        den längsten einfachen Pfad zu finden, ist ein NP-schweres Problem.
        Es ist also nicht in Polynomialzeit zu lösen, wenn man davon ausgeht,
        dass $P \neq NP$ gilt.
        Nun kann man das Longest Path Problem auf das Shortest Simple Path
        Problem reduzieren.
        Der längste einfache Pfad in einem Graphen $G$ entspricht dem kürzesten
        einfachen Pfad in $G'$, wenn $G'$ aus $G$ durch Negation der
        Kantengewichte hervorgeht (in linearer Zeit möglich).
        Hätten wir einen effizienten Algorithmus für unser Problem, so könnten
        wir mit diesem das Longest Path Problem effizient lösen, was in
        Widerspruch zu der Tatsache steht, dass es NP-schwierig ist.


    \item
        Unter der Voraussetzung, dass $G$ keine negativen Zyklen enthält, kann
        das Problem gelöst werden, indem einfach \textsc{BellmanFord} angewandt
        wird.
        Der Algorithmus bestimmt die Länge des kürzesten Pfades von einem
        gegebenen Knoten $s$ zu allen anderen Knoten.
        Das Durchlaufen von positiven Zyklen würde die Länge des Pfades nicht
        verkürzen und negative Zyklen sind nach Voraussetzung nicht vorhanden.
        Also findet \textsc{BellmanFord} alle kürzesten einfachen Pfade, die von
        $s$ ausgehen.

\end{enumerate}

\end{document}
