\documentclass[a4paper]{scrartcl}

% font/encoding packages
\usepackage[utf8]{inputenc}
\usepackage[T1]{fontenc}
\usepackage{lmodern}
\usepackage[ngerman]{babel}

% math
\usepackage{amsmath, amssymb, amsfonts, amsthm, mathtools}
\allowdisplaybreaks
\newtheorem*{behaupt}{Behauptung}
\usepackage{siunitx}

% tikz
\usepackage{tikz}
\usetikzlibrary{graphs}
\usetikzlibrary{arrows}
\usetikzlibrary{positioning}
\usepackage{pgfplots}
\usepgfplotslibrary{fillbetween}

% misc
\usepackage{enumerate}
\usepackage{algorithm}
\usepackage{algpseudocode}
\usepackage{listings}
\usepackage[section]{placeins}
                    
% macros
\newcommand{\linf}[1]{\lim_{#1 \to \infty}}
\newcommand{\gdw}{\Leftrightarrow}
\newcommand{\dif}{\mathop{}\!\mathrm{d}}
\newcommand{\bigo}{\mathcal{O}}

% last package
\usepackage{hyperref}


\title{Algorithmen und Datenstrukturen}
\subtitle{Übungsblatt 5 - Aufgabe 3}
\author{
    anonymous
}
\date{zum 16. Dezember 2014}

\begin{document}
\maketitle

\begin{enumerate}[(a)]
    \item
        Das Problem lässt sich, wie folgt, als gerichteten Graphen $G = (V, E)$
        darstellen.
        Jeder Punkt $(x_i, f(x_i))$ für $i = 1,  \ldots, n$ wird als Knoten
        $i \in V$ repräsentiert.
        Die Menge der Kanten ist definiert als
        \begin{equation}
            E = \left\{ (i, j) \ \vert\  \forall i, j \in V : i < j \right\}
            \text{ .}
        \end{equation}
        Es handelt sich offensichtlich um einen gerichteten, azyklischen Graphen
        (DAG) mit der einzigen topologischen Sortierung
        \begin{equation}
            \pi = 1, 2, 3, \ldots, n
        \end{equation}
        Sei $e$ die Anzahl der durchlaufenden Kanten in einem Pfad.
        $e$ entspricht der Anzahl linearer Segmente von $g$.
        Würde man $g$ durch alle gegebenen Punkte legen, so würde man alle
        $n$ Knoten ablaufen und dabei $n-1$ Kanten durchlaufen.
        „Überspringt“ man $k$ Punkte von $f$, so nimmt man die Kante von $i$
        nach $j = i+k+1$.
        Jede Kante in $E$ muss als Kantengewichtung die Summe der Fehler haben,
        welche zwischen $g$ und $f$ an allen übersprungenen Punkten entstehen.
        Der gegebene Ausdruck wird minimiert, wenn die Summe aus der Anzahl der
        Kanten (mit $\alpha$ gewichtet) und der Summe der Fehler (mit $\beta$
        gewichtet) minimiert wird.
        
        \begin{algorithm}
            \caption{\textsc{DAG-Bellman-Ford}}
            \label{alg:dag-bf}
            \begin{algorithmic}[1]
                \Procedure{Error}{$u, v$}
                    \State $diffquot \gets \frac{f(x_v) - f(x_u)}{x_v - x_u}$
                    \State $error \gets 0$
                    \For {$w \in \left\{ w \in V \ \vert\  w > u \land w < v \right\}$}
                        \State $g(x_w) \gets f(x_u) + diffquot \cdot (x_w - x_u)$
                        \State $error \gets error + \left\vert f(x_w) - g(x_w) \right\vert^2$
                    \EndFor
                    \State \Return $error$
                \EndProcedure
                \Statex
                %
                \Procedure{InitializeSingleSource}{$G, s$}
                    \For {\textbf{all} $v \in V$}
                        \State $v.value \gets \infty$
                        % \State $v.error \gets \infty$
                        % \State $v.edges \gets \infty$
                        \State $v.path \gets None$
                    \EndFor
                    \State $s.value \gets 0$
                    % \State $s.error \gets 0$
                    % \State $s.edges \gets 0$
                    \State $s.path \gets [1,]$
                \EndProcedure
                \Statex
                %
                \Procedure{Relax}{$u, v, \alpha, \beta$}
                    \State $newval \gets u.value + \alpha + \beta \cdot $ \Call{Error}{$u, v$}
                    \If {$u.value > newval$}
                        \State $v.value \gets newval$
                        % \State $v.error \gets u.error + $ \Call{Error}{$u, v$}
                        % \State $v.edges \gets u.edges + 1$
                        \State $v.path \gets u.path + v$
                    \EndIf
                \EndProcedure
                \Statex
                %
                \Procedure{DAG-Bellman-Ford}{$G, s, \alpha, \beta$}
                    \State \Call{InitializeSingleSource}{$G, 1$}
                    \For {$v$ \textbf{in} $1, 2, 3, \ldots, n$}
                        \For {\textbf{all} $(u, v)$ \textbf{in} $G.E$}
                            \State \Call{Relax}{$u, v, \alpha, \beta$}
                        \EndFor
                    \EndFor
                \EndProcedure
            \end{algorithmic}
        \end{algorithm}

    \item

\end{enumerate}

\end{document}
