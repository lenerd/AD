\documentclass[a4paper]{scrartcl}

% font/encoding packages
\usepackage[utf8]{inputenc}
\usepackage[T1]{fontenc}
\usepackage{lmodern}
\usepackage[ngerman]{babel}

% math
\usepackage{amsmath, amssymb, amsfonts, amsthm, mathtools}
\allowdisplaybreaks
\newtheorem*{behaupt}{Behauptung}
\usepackage{siunitx}

% misc
\usepackage{enumerate}
\usepackage{algorithm}
\usepackage{algpseudocode}
\usepackage[section]{placeins}

% tikz
\usepackage{tikz}
\usetikzlibrary{graphs}
\usetikzlibrary{arrows}
\usetikzlibrary{positioning}
\usepackage{pgfplots}
\usepgfplotslibrary{fillbetween}
                    
% macros
\newcommand{\linf}[1]{\lim_{#1 \to \infty}}
\newcommand{\gdw}{\Leftrightarrow}
\newcommand{\dif}{\mathop{}\!\mathrm{d}}
\newcommand{\bigo}{\mathcal{O}}


\title{Algorithmen und Datenstrukturen}
\subtitle{Übungsblatt 5 - Aufgabe 1}
\author{
    anonymous
}
\date{zum 16. Dezember 2014}

\begin{document}
\maketitle

Sei $G = (V, E)$ ein DAG und $s \in V$.
Sei \textbf{vor} die Relation in der topologischen Sortierung.

\begin{algorithm}
    \caption{\textsc{DAG-Bellman-Ford}}
    \label{alg:dag-bf}
    \begin{algorithmic}[1]
        \Procedure{DAG-Bellman-Ford}{$G, s$}
            \State \Call{InitializeSingleSource}{$G, s$}
            \State $list \gets$ \Call{Topological-Sort}{$G$}
            \State $start \gets 0$
            \While {$list[start] \neq s$}
                \State $start \gets start + 1$
            \EndWhile
            \For {$v$ \textbf{in} $list[start + 1 : |V|]$}
                \For {\textbf{all} $(u, v)$ \textbf{in} $G.E$}
                    \State \Call{Relax}{$u, v$}
                \EndFor
            \EndFor
        \EndProcedure
    \end{algorithmic}
\end{algorithm}

\begin{behaupt}
    In Algorithmus \ref{alg:dag-bf} wird \textsc{Relax} höchstens $|E|$-mal
    aufgerufen.
\end{behaupt}
\begin{proof}
    Nach Zeile 3 befinden sich alle Knoten von $G$ in topologischer Reihenfolge
    in $list$.
    Die innere Schleife wird für jeden Knoten $v$ mit $s \textbf{ vor } v$
    einmal aufgerufen.
    Dies sind höchstens $|V| - 1$ Knoten.
    In dieser Schleife wird \textsc{Relax} einmal für jede Kante, welche zu
    diesem Knoten führt, aufgerufen.
    Da jede Kante zu genau einem Knoten führt, wird \textsc{Relax} für jede
    Kante höchstens einmal aufgerufen.
    Dies sind höchstens $|E|$ Ausführungen von \textsc{Relax}.
\end{proof}

\begin{behaupt}
    Algorithmus \ref{alg:dag-bf} löst das Single-Source-Shortest-Path-Problem.
\end{behaupt}
\begin{proof}
    Sei $t \in V$.
    \begin{itemize}
        \item $t = s$ \\
            In \textsc{InitializeSingleSource} wird die korrekte Distanz (0) gesetzt.

        \item $t \textbf{ vor } s$ \\
            Da $G$ ein DAG ist und es einen Pfad von $t$ nach $s$ gibt,
            existiert kein Pfad von $s$ zu $t$ (sonst wäre $G$ nicht azylisch).
            $t$ steht vor $s$ in $list$.
            Da nur der Teil von $list$ ab $s$ bearbeitet wird, wird die Distanz
            von $s$ nach $t$ nicht neu gesetzt, nachdem sie mit $\infty$
            initialisiert wurde.

        \item $t \neq s \land \lnot (t \textbf{ vor } s) \land \lnot (s \textbf{ vor } t)$ \\
            Es gibt keinen Pfad zwischen $s$ und $t$.
            Falls $t$ vor $s$ in $list$ steht, wird die Schleife nicht auf $t$
            ausgeführt, sonst wird wie in dem nächsten Punkt vorgegangen.

        \item $s \textbf{ vor } t$
            \begin{enumerate}
                \item Induktionsanfang: $(s, t) \in E$ \\
                    Sobald die äußere Schleife bei $t$ angekommen ist, wird
                    \textsc{Relax} für diese Kante aufgerufen und setzt die
                    korrekte Distanz von $s$ zu $t$ (gegeben durch $w(s, t)$),
                    da $t.dist = \infty > s.dists + w(s, t) = w(s, t)$ galt.

                \item Induktionsannahme: Bei Erreichen eines Knotens $k$ in
                    $list$ wird die korrekte Distanz zu $s$ gesetzt.

                \item Induktionsschritt $(s, t) \notin E$ \\
                    Dann gibt es eine Menge von Knoten
                    \begin{equation}
                        K = \left\{ k \in V \ |\  s \textbf{ vor } k \land (k, t) \in E \right\}
                    \end{equation}
                    Da für alle $k \in K$ gilt, $(k, t) \in E$, gilt auch $k \textbf{ vor } t$.
                    Bevor die äußere Schleife $t$ erreicht, hat sie bereits alle
                    $k$ bearbeitet.
                    Nach der Induktionsannahme haben alle $k$ bereits die
                    korrekte Distanz zu $s$.
                    Während $t$ in der Schleife gewählt ist, werden die Kanten
                    von $k \in K$ durchlaufen und
                    $t.dist = \textrm{min}(k.dist + w(k, t))$ gesetzt.
                    Ist $K = \emptyset$, so bleibt $t.dist = \infty$.
                    
            \end{enumerate}

    \end{itemize}
\end{proof}

\end{document}
