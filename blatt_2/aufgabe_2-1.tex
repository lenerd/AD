\documentclass[a4paper]{scrartcl}

% font/encoding packages
\usepackage[utf8]{inputenc}
\usepackage[T1]{fontenc}
\usepackage{lmodern}
\usepackage[ngerman]{babel}

% math
\usepackage{amsmath, amssymb, amsfonts, amsthm, mathtools}
\allowdisplaybreaks
\newtheorem*{behaupt}{Behauptung}
\usepackage{siunitx}

% tikz
\usepackage{tikz}
\usetikzlibrary{graphs}
\usetikzlibrary{arrows}
\usetikzlibrary{positioning}
\usepackage{pgfplots}
\usepgfplotslibrary{fillbetween}

% misc
\usepackage{enumerate}
\usepackage{algorithm}
\usepackage{algpseudocode}
\usepackage{listings}
\usepackage[section]{placeins}
                    
% macros
\newcommand{\linf}[1]{\lim_{#1 \to \infty}}
\newcommand{\gdw}{\Leftrightarrow}
\newcommand{\dif}{\mathop{}\!\mathrm{d}}
\newcommand{\bigo}{\mathcal{O}}

% last package
\usepackage{hyperref}


\title{Algorithmen und Datenstrukturen}
\subtitle{Übungsblatt 1 - Aufgabe 1}
\author{
%    anonymous
}
\date{zum 21. Oktober 2014}

\begin{document}
\maketitle

\begin{enumerate}
    \item
        \begin{enumerate}[(a)]
            \item
            Wieviele Kanten hat ein k-närer Baum der Tiefe ` mit insgesamt n Knoten?\\
            
			Ein k-närer Baum hat \textit{n-1} Knoten, da jeder Knoten außer der Wurzelknoten eine Kante 
			zum Elternknoten hat.\\
            
            \item
            Wieviele Knoten liegen maximal in Level \textit{l} >= 0 ?\\
            
            Level \textit{l} = 0 -> 1 Knoten, Level \textit{l} = 1 -> maximal 1 * k = $k^1$ Knoten,\\
            Level \textit{l} = 2 -> maximal 1 * k * k = $k^2$ Knoten, usw., also sind es in Level \textit{l}
            maximal $k^l$ Knoten.\\
            
            \item
            Wieviele Knoten hat ein voller Baum der Tiefe \textit{l} insgesamt?\\
            
            Summer aller Ebenen bei einem vollen Baum $\sum_{i=0}^n c^i$ = $\frac{k^{l+1}-1}{k-1}$ *\\
            (*) harmonische Reihe für q != 1\\
            
            \item
            Wieviele Knoten hat ein vollständiger Baum der Tiefe \textit{l} insgesamt?\\
			
			Im letzten Level des vollständigen Baums muss mindestens 1 Knoten sein, was bedeutet, 
			dass bis zu $k^{l-1}-1$ Knoten fehlen dürfen. \\
			D.h. es gibt $\frac{k^{l+1}-1}{k-1}-(k^{l}-1)$ = $\frac{k^{l}-1}{k-1}+1$ Knoten.
        \end{enumerate}
        
        
\end{enumerate}


\end{document}
