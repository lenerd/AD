\documentclass[a4paper]{scrartcl}

% font/encoding packages
\usepackage[utf8]{inputenc}
\usepackage[T1]{fontenc}
\usepackage{lmodern}
\usepackage[ngerman]{babel}

% math
\usepackage{amsmath, amssymb, amsfonts, amsthm, mathtools}
\allowdisplaybreaks
\newtheorem*{behaupt}{Behauptung}
\usepackage{siunitx}

% tikz
\usepackage{tikz}
\usetikzlibrary{graphs}
\usetikzlibrary{arrows}
\usetikzlibrary{positioning}
\usepackage{pgfplots}
\usepgfplotslibrary{fillbetween}

% misc
\usepackage{enumerate}
\usepackage{algorithm}
\usepackage{algpseudocode}
\usepackage{listings}
\usepackage[section]{placeins}
                    
% macros
\newcommand{\linf}[1]{\lim_{#1 \to \infty}}
\newcommand{\gdw}{\Leftrightarrow}
\newcommand{\dif}{\mathop{}\!\mathrm{d}}
\newcommand{\bigo}{\mathcal{O}}

% last package
\usepackage{hyperref}


\title{Algorithmen und Datenstrukturen}
\subtitle{Übungsblatt 2 - Aufgabe 1}
\author{
    anonymous
}
\date{zum 4. November 2014}

\begin{document}
\maketitle

Es werden $k$-näre Bäume betrachtet.
\begin{enumerate}[(a)]
    \item
        Ein Baum mit insgesamt $n$ Knoten besitzt $n-1$ Kanten.
        
    \item
        In Level $l \geq 0$ liegen maximal $k^l$ Knoten.
        
    \item
        Wir gehen davon aus, dass der Begriff \emph{full} aus den Folien der
        Vorlesungen mit \emph{voll} übersetzt wurde.

        Ein voller Baum der Tiefe $l$ hat $\frac{k^{l+1}-1}{k-1}$ Knoten.
        
    \item
        Wir gehen davon aus, dass der Begriff \emph{complete} aus den Folien der
        Vorlesungen mit \emph{vollständig} übersetzt wurde.

        Ein vollständiger Baum der Tiefe $l$ hat $n \in \left[ \frac{k^l-1}{k-1}+1 , \frac{k^{l+1}-1}{k-1} \right]$ Knoten.
        
\end{enumerate}


\end{document}
