\documentclass[a4paper]{scrartcl}

% font/encoding packages
\usepackage[utf8]{inputenc}
\usepackage[T1]{fontenc}
\usepackage{lmodern}
\usepackage[ngerman]{babel}

% math
\usepackage{amsmath, amssymb, amsfonts, amsthm, mathtools}
\allowdisplaybreaks
\newtheorem*{behaupt}{Behauptung}
\usepackage{siunitx}

% misc
\usepackage{enumerate}
\usepackage{algorithm}
\usepackage{algpseudocode}
\usepackage[section]{placeins}

% tikz
\usepackage{tikz}
\usetikzlibrary{graphs}
\usetikzlibrary{arrows}
\usetikzlibrary{positioning}
\usepackage{pgfplots}
\usepgfplotslibrary{fillbetween}
                    
% macros
\newcommand{\linf}[1]{\lim_{#1 \to \infty}}
\newcommand{\gdw}{\Leftrightarrow}
\newcommand{\dif}{\mathop{}\!\mathrm{d}}
\newcommand{\bigo}{\mathcal{O}}


\title{Algorithmen und Datenstrukturen}
\subtitle{Übungsblatt 2 - Aufgabe 3}
\author{
    anonymous
}
\date{zum 4. November 2014}

\begin{document}
\maketitle

\begin{enumerate}[(a)]
    \item
        Es werden die üblichen Operationen eines Stacks vorrausgesetzt
        (push, pop, isempty, \ldots).

        Alle Elemente werden bei \textsc{Enqueue} zunächst auf den Stack $Q.A$
        gelegt.
        Bei \textsc{Dequeue} wird das oberste Element von $Q.B$ zurückgegeben.
        Ist dieser leer, so werden zuvor die Elemente von $Q.A$ auf $Q.B$
        umgeschichtet.
        In der Folge wird die Reihenfolge umgekehrt, sodass die gewünschte
        FIFO-Eigenschaft erfüllt wird.
        Jedes Element, befindet sich jeweils genau einmal auf jedem Stack.

        \begin{algorithm}
            \caption{\textsc{Enqueue}$(Q, val)$}
            \begin{algorithmic}[1]
            \Require Eine Queue $Q$ mit den Stacks $Q.A$ und $Q.B$.
                Ein Wert $val$.
            \Ensure $val$ wird ans Ende der Queue angefügt.
            \State push($val$, $Q.A$)
            \end{algorithmic}
        \end{algorithm}

        \begin{algorithm}
            \caption{\textsc{Dequeue}$(Q)$}
            \begin{algorithmic}[1]
            \Require Eine nicht leere Queue $Q$ mit den Stacks $Q.A$ und $Q.B$.
            \Ensure Das vorderste Element der Queue wird zurückgegeben.
            \If{isempty($Q.B$)}
                \While{not isempty($Q.A$)}
                    \State push($Q.B$, pop($Q.A$))
                \EndWhile
            \EndIf
            \State \Return pop($Q.B$)
            \end{algorithmic}
        \end{algorithm}
        Da \textsc{Enqueue} aus einer einzigen Operation besteht, gilt
        \begin{equation}
            \textsc{Enqueue} \in \mathcal{O}\left( 1 \right) \text{ .}
        \end{equation}
        Im Worst-Case, befinden sich $n$ Elemente auf $Q.A$ und $Q.B$ ist leer.
        Bei \textsc{Dequeue} müssen alle $n$ Elemente auf den zweiten Stack
        verschoben werden.
        Die Funktion ist also linear von $n$ abhängig:
        \begin{equation}
            \textsc{Denqueue} \in \mathcal{O}\left( n \right) \text{ .}
        \end{equation}
        
    \item
        Der Worst-Case für die Funktionsfolge wäre, wenn alle $n_E$ Elemente
        auf $Q.B$ geschoben werden (z.\,B. $n_E = n-1$, $n_D = 1$: Zunächst
        werden $n_E$ \textsc{Enqueue}, dann eine \textsc{Dequeue} ausgeführt.)
        \begin{align*}
            T_n \in \mathcal{O}(n) \\
            \frac{T_n}{n} \in \mathcal{O}(1)
        \end{align*}
        
        

\end{enumerate}


\end{document}
