\documentclass[a4paper]{scrartcl}

% font/encoding packages
\usepackage[utf8]{inputenc}
\usepackage[T1]{fontenc}
\usepackage{lmodern}
\usepackage[ngerman]{babel}

% math
\usepackage{amsmath, amssymb, amsfonts, amsthm, mathtools}
\allowdisplaybreaks
\newtheorem*{behaupt}{Behauptung}
\usepackage{siunitx}

% tikz
\usepackage{tikz}
\usetikzlibrary{graphs}
\usetikzlibrary{arrows}
\usetikzlibrary{positioning}
\usepackage{pgfplots}
\usepgfplotslibrary{fillbetween}

% misc
\usepackage{enumerate}
\usepackage{algorithm}
\usepackage{algpseudocode}
\usepackage{listings}
\usepackage[section]{placeins}
                    
% macros
\newcommand{\linf}[1]{\lim_{#1 \to \infty}}
\newcommand{\gdw}{\Leftrightarrow}
\newcommand{\dif}{\mathop{}\!\mathrm{d}}
\newcommand{\bigo}{\mathcal{O}}

% last package
\usepackage{hyperref}


\title{Algorithmen und Datenstrukturen}
\subtitle{Übungsblatt 2 - Aufgabe 4}
\author{
    anonymous
}
\date{zum 4. November 2014}

\begin{document}
\maketitle

\begin{enumerate}[(a)]
    \item Sei n eine fest gewählte Zahl unf $D(f) = \mathbb{R}_{>1}$.
        \begin{align}
            f(x) &= x \log_x (n) = \ln(n) \cdot \frac{x}{\ln(x)} \\
            \frac{\dif}{\dif x} f(x) &= \ln(n) \cdot \frac{\ln(x)-1}{\ln^2(x)} \\
            \frac{\dif}{\dif x} f(e) &= \ln(n) \cdot \frac{\ln(e)-1}{\ln^2(e)}
            = \ln(n) \cdot \frac{1-1}{1} = 0 \\
            \frac{\dif^2}{\dif x^2} f(x) &= \ln(n) \cdot \frac{-\ln(x)+2}{x \cdot \ln^3(x)} \\
            \frac{\dif^2}{\dif x^2} f(e) &= \ln(n) \cdot \frac{-\ln(e)+2}{e \cdot \ln^3(e)}
            = \ln(n) \cdot \frac{1}{e} > 0
        \end{align}
        $f$ besitzt ein Minimum bei $x=e$.

    \item Plot von $f$ mit $n=2$
        \begin{center}
            \begin{tikzpicture}
                \begin{axis}[
                    axis lines=middle,
                    axis equal,
                    xlabel=$x_1$,
                    ylabel=$x_2$,
                    width=1.0\textwidth,
                    xmin = 0, xmax = 10,
                    ymin = 0, ymax = 10,
                    x axis line style={name path=xaxis}
                ]
                    \addplot
                    [name path=f, samples=200, domain=0:10]
                    {ln(2)*x/ln(x)};
                \end{axis}
            \end{tikzpicture}
        \end{center}
        Da das Minimum bei $e$ liegt, wählen wir $\lceil e \rceil = 3$ als $k^*$. \\
        \begin{tabular}{r|r|r}
            $n$    & $\lceil k \log_k (n) \rceil, k=2$ & $\lceil k \log_k (n) \rceil, k^* = 3$ \\ \hline
            $10^1$ &\num{7}  &\num{7}  \\
            $10^2$ &\num{14} &\num{13} \\
            $10^3$ &\num{20} &\num{19} \\
            $10^4$ &\num{27} &\num{26} \\
            $10^5$ &\num{34} &\num{32} \\
            $10^6$ &\num{40} &\num{38} \\
            $10^7$ &\num{47} &\num{45} \\
            $10^8$ &\num{54} &\num{51} \\
            $10^9$ &\num{60} &\num{57} \\
        \end{tabular}

    \item
        % In einem $k$-nären Heap mit $n$ Elementen benötigt die
        % \textsc{Heapify}-Operation
        % \begin{equation}
        %     T(n) = \Big\lceil \underbrace{k}_{(1)} \cdot \underbrace{\log_k (n)}_{(2)} \Big\rceil
        %     \text{ Schritte.}
        % \end{equation}
        % Wobei $T_n$ sich wie folgt zusammensetzt:
        % \begin{enumerate}[(1)]
        %     \item Die Anzahl der Kinder.
        %     \item Die Anzahl der Elemente.
        % \end{enumerate}
        % Verwaltet jeder Knoten seine Kinder in einem binären Heap, so muss (1)
        % durch die Anzahl der Schritte zum Finden (und ggf. Vertauschen) des
        % Maximums des binären Heaps ersetzt werden.

    \item 
        \begin{enumerate}[(i)]
            \item
                Nach dem Ersetzen von 9 durch 1 werden zwei Vertauschungen
                ($(1,8), (1,7)$) durchgeführt. 

                Als Bäume: \medskip \\
                \begin{tikzpicture}[
                        level 1/.style={sibling distance=4em},
                        level 2/.style={sibling distance=2em},
                    ]
                    \begin{scope}
                        \node {9} % root
                            child { node {8}
                                child { node {7} }
                                child { node {3} }
                            }
                            child { node {6}
                                child { node {5} }
                                child { node {2} }
                            };
                    \end{scope}
                    \begin{scope}[shift={(3,0)}]
                        \node (a) {\underline{1}} % root
                            child { node (b) {8}
                                child { node {7} }
                                child { node {3} }
                            }
                            child { node {6}
                                child { node {5} }
                                child { node {2} }
                            };
                        \draw[<->] (a) to [bend right] (b);
                    \end{scope}
                    \begin{scope}[shift={(6,0)}]
                        \node {8} % root
                            child { node (a) {\underline{1}}
                                child { node (b) {7} }
                                child { node {3} }
                            }
                            child { node {6}
                                child { node {5} }
                                child { node {2} }
                            };
                        \draw[<->] (a) to [bend right] (b);
                    \end{scope}
                    \begin{scope}[shift={(9,0)}]
                        \node {8} % root
                            child { node {7}
                                child { node {\underline{1}} }
                                child { node {3} }
                            }
                            child { node {6}
                                child { node {5} }
                                child { node {2} }
                            };
                    \end{scope}
                \end{tikzpicture}

                In Level-Order: \medskip \\
                \begin{tabular}{|l|c|c|c|c|c|c|c|}
                    \hline
                   $t = 0$ & 9 & 8 & 6 & 7 & 3 & 5 & 2 \\ \hline
                   $t = 1$ & 1 & 8 & 6 & 7 & 3 & 5 & 2 \\ \hline
                   $t = 2$ & 8 & 1 & 6 & 7 & 3 & 5 & 2 \\ \hline
                   $t = 3$ & 8 & 7 & 6 & 1 & 3 & 5 & 2 \\ \hline
                \end{tabular}

            \item
                Nach dem Ersetzen von 9 durch 1 wird eine Vertauschung
                ($(1,8)$) durchgeführt. 

                Als Bäume: \medskip \\
                \begin{tikzpicture}[
                        level 1/.style={sibling distance=2em},
                        level 2/.style={sibling distance=2em},
                    ]
                    \begin{scope}
                        \node {9} % root
                            child { node {7}
                                child { node {3} }
                                child { node {6} }
                                child { node {2} }
                            }
                            child { node {5} }
                            child { node {8} };
                    \end{scope}
                    \begin{scope}[shift={(3,0)}]
                        \node (a) {\underline{1}} % root
                            child { node {7}
                                child { node {3} }
                                child { node {6} }
                                child { node {2} }
                            }
                            child { node {5} }
                            child { node (b) {8} };
                        \draw[<->] (a) to [bend left] (b);
                    \end{scope}
                    \begin{scope}[shift={(6,0)}]
                        \node {8} % root
                            child { node {7}
                                child { node {3} }
                                child { node {6} }
                                child { node {2} }
                            }
                            child { node {5} }
                            child { node {\underline{1}} };
                    \end{scope}
                    \begin{scope}[shift={(9,0)}]
                    \end{scope}
                \end{tikzpicture}

                In Level-Order: \medskip \\
                \begin{tabular}{|l|c|c|c|c|c|c|c|}
                    \hline
                    $t = 0$ & 9 & 7 & 5 & 8 & 3 & 6 & 2 \\ \hline
                    $t = 1$ & 1 & 7 & 5 & 8 & 3 & 6 & 2 \\ \hline
                    $t = 2$ & 8 & 7 & 5 & 1 & 3 & 6 & 2 \\ \hline
                \end{tabular}
        \end{enumerate}

    \item

\end{enumerate}


\end{document}
