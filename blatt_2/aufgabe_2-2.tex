\documentclass[a4paper]{scrartcl}

% font/encoding packages
\usepackage[utf8]{inputenc}
\usepackage[T1]{fontenc}
\usepackage{lmodern}
\usepackage[ngerman]{babel}

% math
\usepackage{amsmath, amssymb, amsfonts, amsthm, mathtools}
\allowdisplaybreaks
\newtheorem*{behaupt}{Behauptung}
\usepackage{siunitx}

% tikz
\usepackage{tikz}
\usetikzlibrary{graphs}
\usetikzlibrary{arrows}
\usetikzlibrary{positioning}
\usepackage{pgfplots}
\usepgfplotslibrary{fillbetween}

% misc
\usepackage{enumerate}
\usepackage{algorithm}
\usepackage{algpseudocode}
\usepackage{listings}
\usepackage[section]{placeins}
                    
% macros
\newcommand{\linf}[1]{\lim_{#1 \to \infty}}
\newcommand{\gdw}{\Leftrightarrow}
\newcommand{\dif}{\mathop{}\!\mathrm{d}}
\newcommand{\bigo}{\mathcal{O}}

% last package
\usepackage{hyperref}


\title{Algorithmen und Datenstrukturen}
\subtitle{Übungsblatt 2 - Aufgabe 2}
\author{
    anonymous
}
\date{zum 4. November 2014}

\begin{document}
\maketitle

\begin{enumerate}[(a)]
    \item Sei $k \in \{ 1,2,3 \}$ und $T_k(n)$ die Laufzeit für
        $\textsc{Order}k(v)$.
        Da nicht gesagt wird, dass es sich um einen vollständigen, binären Baum
        handelt, kann das Master-Theorem nicht angewandt werden.
        
        Der Baum wird rekursiv durchlaufen, dabei wird jeder Knoten einmal
        besucht. Da der Baum $n$ Knoten besitzt, wird \textsc{Print} $n$-mal
        aufgerufen.

        \begin{equation}
            \Rightarrow T_k(n) \in \mathcal{O} \left( n \right)
        \end{equation}
        

    \item Die best-case Laufzeit liegt in $\Omega \left( n \right)$ und damit
        auch in $\Theta \left( n \right)$.
        Es müssen in jedem Fall alle Kanten zu allen Knoten durchgelaufen
        werden. Dies sind $n-1$ Stück.

    \item Die Ausgaben sind \\
        \textsc{Order1}: \texttt{SRITEHGIEMOLWKAL}, \\
        \textsc{Order2}: \texttt{ETIHRIGESLOWMAKL}, \\
        \textsc{Order3}: \texttt{ETHIIEGRLWOALKMS}.

    \item
        \begin{tabular}{|c|c|c|c|c|c|c|c|c|c|}
            \hline
            S & V & T & N & R & I & Y & U & I & E \\
            \hline
        \end{tabular}

    \item Die Ausgabe ist \\
        \texttt{WELIKEALGORITHMS}.

\end{enumerate}


\end{document}
