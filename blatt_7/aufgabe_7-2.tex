\documentclass[a4paper]{scrartcl}

% font/encoding packages
\usepackage[utf8]{inputenc}
\usepackage[T1]{fontenc}
\usepackage{lmodern}
\usepackage[ngerman]{babel}

% math
\usepackage{amsmath, amssymb, amsfonts, amsthm, mathtools}
\allowdisplaybreaks
\newtheorem*{behaupt}{Behauptung}
\usepackage{siunitx}

% tikz
\usepackage{tikz}
\usetikzlibrary{graphs}
\usetikzlibrary{arrows}
\usetikzlibrary{positioning}
\usepackage{pgfplots}
\usepgfplotslibrary{fillbetween}

% misc
\usepackage{enumerate}
\usepackage{algorithm}
\usepackage{algpseudocode}
\usepackage{listings}
\usepackage[section]{placeins}
                    
% macros
\newcommand{\linf}[1]{\lim_{#1 \to \infty}}
\newcommand{\gdw}{\Leftrightarrow}
\newcommand{\dif}{\mathop{}\!\mathrm{d}}
\newcommand{\bigo}{\mathcal{O}}

% last package
\usepackage{hyperref}


\title{Algorithmen und Datenstrukturen}
\subtitle{Übungsblatt 7 - Aufgabe 2}
\author{
    anonymous
}
\date{zum 27. Januar 2015}

\begin{document}
\maketitle

\begin{enumerate}[(a)]
    \item 
        Die gewählte Strategie sortiert $A$ und $R$ absteigend, so dass
        $a_i \geq a_{i+1}$ und $r_i \geq r_{i+1}$ für alle $i$ gilt.
        Da die Multiplikation kommutativ ist, kann auch aufsteigend sortiert
        werden.

    \item
        \begin{behaupt}
            Die genannte Strategie für zu dem maximalen Wert.
        \end{behaupt}
        \begin{proof}
            Seien $1 \leq i < j \leq n$.
            Es muss gezeigt werden, dass die sortierte Reihenfolge mindestens
            ein genauso gutes Ergebnis liefert, wie andere Verfahren.
            \begin{equation}
                a_i^{r_i} \cdot a_j^{r_j} \geq a_i^{r_j} \cdot a_j^{r_i}
                \label{eq:geq}
            \end{equation}
            Aus der Strategie folgen $a_i \geq a_j$ und $r_i \geq r_j$.
            Es gilt also
            \begin{equation}
                r_i - r_j \geq 0
                \text{ und }
                \forall x \geq 0 : a_i^x \geq a_j^x
                \text{ .}
            \end{equation}
            \begin{equation}
                \begin{split}
                    a_i^{r_i - r_j} &\geq a_j^{r_i - r_j} \\
                    \gdw a_i^{r_j} a_j^{r_j} \cdot a_i^{r_i - r_j}
                    &\geq a_i^{r_j} a_j^{r_j} \cdot a_j^{r_i - r_j} \\
                    \gdw a_i^{r_i} a_j^{r_j} &\geq a_i^{r_j} a_j^{r_i}
                \end{split}
            \end{equation}
            Gleichung \eqref{eq:geq} gilt für ein beliebiges Paar von Indizes
            $i < j$ und auch wenn die Reste der Folgen identisch sind, da auf
            beiden Seiten ein konstanter Faktor hinzugefügt wird.


            

            
        \end{proof}

\end{enumerate}

\end{document}
