\documentclass[a4paper]{scrartcl}

% font/encoding packages
\usepackage[utf8]{inputenc}
\usepackage[T1]{fontenc}
\usepackage{lmodern}
\usepackage[ngerman]{babel}

% math
\usepackage{amsmath, amssymb, amsfonts, amsthm, mathtools}
\allowdisplaybreaks
\newtheorem*{behaupt}{Behauptung}
\usepackage{siunitx}

% tikz
\usepackage{tikz}
\usetikzlibrary{graphs}
\usetikzlibrary{arrows}
\usetikzlibrary{positioning}
\usepackage{pgfplots}
\usepgfplotslibrary{fillbetween}

% misc
\usepackage{enumerate}
\usepackage{algorithm}
\usepackage{algpseudocode}
\usepackage{listings}
\usepackage[section]{placeins}
                    
% macros
\newcommand{\linf}[1]{\lim_{#1 \to \infty}}
\newcommand{\gdw}{\Leftrightarrow}
\newcommand{\dif}{\mathop{}\!\mathrm{d}}
\newcommand{\bigo}{\mathcal{O}}

% last package
\usepackage{hyperref}


\title{Algorithmen und Datenstrukturen}
\subtitle{Übungsblatt 7 - Aufgabe 1}
\author{
    anonymous
}
\date{zum 27. Januar 2015}

\begin{document}
\maketitle

\begin{enumerate}[(a)]
    \item \hfill \\
        \begin{algorithm}
            \caption{\textsc{Max Non Adjacent Sum}}
            \label{alg:1}
            \begin{algorithmic}[1]
                \Procedure{MNAS}{$a$}
                \State $m \gets$ new array of size $n$
                \State $m_1 \gets a_1$
                \If {$a_2 > a_1$}
                    \State $m_2 \gets a_2$
                \Else
                    \State $m_2 \gets a_1$
                \EndIf
                \For {$i = 3, \ldots, n$}
                    \If {$m_{i-2} + a_i > m_{i-1}$}
                        \State $m_i \gets m_{i-2} + a_i$
                    \Else
                        \State $m_i \gets m_{i-1}$
                    \EndIf
                \EndFor
                \State \Call{Print}{$m_n$}
                \EndProcedure
            \end{algorithmic}
        \end{algorithm}
        \paragraph{Korrektheit} \hfill \\
        Der Algorithmus \ref{alg:1} geht davon aus, dass die Folge mindestens
        zwei Elemente besitzt.
        Eine Anpassung für ein- und nullelementige Folgen ist leicht möglich,
        indem der Wert des Elements bzw. 0 zurückgegeben wird.
        Der Algorithmus hat eine Laufzeit in $\mathcal{O}(n)$, da
        ausschließlich die einfache for-Schleife von $n$ abhängig ist.
        \begin{behaupt}
            Der Algorithmus berechnet das maximale Gewicht einer Teilfolge
            einer Folge $a$ von $n$ positiven Zahlee unter der Bedingung, dass
            keine zwei benachbarten Glieder von $a$ in dieser Teilfolge
            vorkommen.
            $m_i$ ist das maximale Gewicht dieser Teilfolge in dem Abschnitt
            $a_1, \ldots, a_i$.
        \end{behaupt}
        \begin{proof}[Beweis durch vollständige Induktion] \hfill \\
            \subparagraph{Induktionsanfang} \hfill \\
            \begin{enumerate}
                \item $i = 1$ \\
                    Da nur ein Element betrachtet wird, gilt $m_1 = a_1$.
                \item $i = 2$ \\
                    Da benachbarte Elemente nicht in der Teilfolge enthalten
                    sein dürfen, gilt $m_2 = \max(a_1, a_2)$
            \end{enumerate}
            \subparagraph{Induktionsschritt} \hfill \\
            Gelte die Behauptung für $i-1$ und $i-2$ ($i$ fest).

            \begin{enumerate}
                \item $a_{i-1}$ ist nicht in der Teilfolge vorhanden. \\
                    Dann gilt $m_{i-1} = m_{i-2}$.
                    Da $a$ nur positive Zahlen enthält, vergrößtert das
                    Hinzufügen von $a_i$ den Wert:
                    $m_i = m_{i-1} + a_i = m_{i-2} + a_i$.

                \item $a_{i-1}$ ist in der Teilfolge vorhanden. \\
                    Es gibt die Möglichkeiten, die Teilfolge mit $a_{i-1}$ zu
                    übernehmen oder die Teilfolge mit $a_{i-2}$ konkateniert
                    mit $a_i$ zu wählen.
                    Da das Gewicht maximiert werden soll wird
                    $m_i = \max(m_{i-2} + a_i, m_{i-1})$ gewählt.
            \end{enumerate}
            In beiden Fällen gilt $m_i = \max(m_{i-2} + a_i, m_{i-1})$, was in
            den Zeilen 10 bis 14 berechnet wird.

            $m_i$ enthält also immer den maximalen Wert einer Teilfolge.
        \end{proof}

    \item \hfill \\
        \begin{algorithm}
            \caption{\textsc{Max Non Adjacent Sum}}
            \label{alg:2}
            \begin{algorithmic}[1]
                \Procedure{MNAS}{$a$}
                \State $m \gets$ new array of size $n$
                \State $l \gets$ new array of size $n$
                \State $m_1 \gets a_1$
                \State $l_1 \gets 1$
                \If {$a_2 > a_1$}
                    \State $m_2 \gets a_2$
                    \State $l_2 \gets 2$
                \Else
                    \State $m_2 \gets a_1$
                    \State $l_2 \gets 1$
                \EndIf
                \For {$i = 3, \ldots, n$}
                    \If {$m_{i-2} + a_i > m_{i-1}$}
                        \Comment $a_i$ wird hinzugefügt
                        \State $m_i \gets m_{i-2} + a_i$
                        \State $l_i \gets i$
                    \Else
                        \Comment $a_i$ wird nicht hinzugefügt
                        \State $m_i \gets m_{i-1}$
                        \State $l_i \gets i - 1$
                    \EndIf
                \EndFor
                \State \Call{Print}{$m_n$}
                \State \Call{Print\_Sequence}{$a, m, l, n$}
                \EndProcedure
                \Procedure{Print\_Sequence}{$a, l, k$}
                \If {$k > 2$}
                    \State \Call{Print\_Sequence}{$a, l, l_k - 2$}
                \EndIf
                \State \Call{Print}{$a_{l_k}$}
                \EndProcedure
            \end{algorithmic}
        \end{algorithm}
        \paragraph{Korrektheit} \hfill \\
        In Algorithmus \ref{alg:2} wurde das Array $l$ eingeführt.
        \begin{behaupt}
            $l_i$ enthält den Index des letzten Folgengliedes der Teilfolge
            maximalen Gewichts in dem Abschnitt $a_1, \ldots, a_i$.
        \end{behaupt}
        \begin{proof}[Beweis durch vollständige Induktion] \hfill \\
            \subparagraph{Induktionsanfang}
            \begin{enumerate}
                \item $i = 1$ \\
                    Offensichtlich ist das einzige Element des Abschnitts auch
                    das letzte Folgenglied: $l_1 = 1$
                \item $i = 2$ \\
                    Es wird das Maximum von $a_1$ und $a_2$ gewählt.
                    Je nach Ergebnis gilt $l_2 = 1$ oder $l_2 = 2$.
            \end{enumerate}

            \subparagraph{Induktionsschritt}
            Gelte die Behauptung für $i-1$ und $i-2$ ($i$ fest).

            \begin{enumerate}
                \item $a_i$ gehört zu der Teilfolge \\
                    $a_i$ ist das letzte Element des Abschnitts.
                    Es folgt $l_i = i$.
                    
                \item $a_i$ gehört nicht zu der Teilfolge \\
                    Dann muss $a_{i-1}$ enthalten sein.
                    Es folgt $l_i = i - 1$.
            \end{enumerate}

            $l_i$ enthält also immer den den Index des letzten Elements einer
            Teilfolge mit maximalen Wert.
        \end{proof}

    \item
        Die Algorithmen funktionieren in dieser Form nicht, wenn negative
        Zahlen in $a$ vorhanden sind.
        Sei $a = -1, -1$.
        Bei der Ausführung würden $m_1 = m_2 = -1$ gesetzt werden, obwohl eine
        leere Teilfolge ein höheres Gewicht ($0$) besitzt.

\end{enumerate}

\end{document}
