\documentclass[a4paper]{scrartcl}

% font/encoding packages
\usepackage[utf8]{inputenc}
\usepackage[T1]{fontenc}
\usepackage{lmodern}
\usepackage[ngerman]{babel}

% math
\usepackage{amsmath, amssymb, amsfonts, amsthm, mathtools}
\allowdisplaybreaks
\newtheorem*{behaupt}{Behauptung}
\usepackage{siunitx}

% misc
\usepackage{enumerate}
\usepackage{algorithm}
\usepackage{algpseudocode}
\usepackage[section]{placeins}

% tikz
\usepackage{tikz}
\usetikzlibrary{graphs}
\usetikzlibrary{arrows}
\usetikzlibrary{positioning}
\usepackage{pgfplots}
\usepgfplotslibrary{fillbetween}
                    
% macros
\newcommand{\linf}[1]{\lim_{#1 \to \infty}}
\newcommand{\gdw}{\Leftrightarrow}
\newcommand{\dif}{\mathop{}\!\mathrm{d}}
\newcommand{\bigo}{\mathcal{O}}


\title{Algorithmen und Datenstrukturen}
\subtitle{Übungsblatt 7 - Aufgabe 3}
\author{
    anonymous
}
\date{zum 27. Januar 2015}

\begin{document}
\maketitle

\section*{Multiway Cut mit lokaler Suche}
Es sei eine Startlösung vorhanden, welche korrekt ist, aber über deren Qualität
keine Aussage gemacht wird.
Aus der gegebenen Menge zu entfernender Kanten $E'$ kann die Menge der
Zusammenhangskomponenten $Z$ des resultierenden Graphen $G(V, E \backslash E')$
gefunden werden.

In der Nachbarschaft dieser Lösung befinden sich alle $Z'$, in denen ein Knoten
$v$ von einer Menge in eine andere gewechselt ist:
\begin{equation}
    \begin{split}
        Z_i' &= Z_i \backslash \{v\} \\
        Z_j' &= Z_j \cup \{v\} \\
        Z_k' &= Z_k \text{ für alle } Z_k \in Z : k \neq i \land k \neq j
    \end{split}
\end{equation}
Bedingung ist, dass die Knoten in allen $Z_i'$ in $G$ jeweils zusammenhängend
sind.
Wird die Anzahl an Kanten, welche für den Multiway Cut entfernt werden müssen,
reduziert, so wird dies als eine bessere Lösung angesehen.

\paragraph{Beispiel (Abb. \ref{fig:local})}
Seien die dunkel gefärbten Knoten $a, c, e \in S$.
Die hell gefärbten Knoten befinden sich in der jeweiligen Teilmenge.
\begin{equation}
    \begin{split}
        Z_a &= \left\{ a, b \right\} \\
        Z_c &= \left\{ c, d \right\} \\
        Z_e &= \left\{ e \right\}
    \end{split}
\end{equation}
Es müssen $|E'| = 4$ Kanten für den Multiway Cut entfernt werden.
\begin{equation}
    E' = \left\{ 
        \left\{ a, e \right\},
        \left\{ b, c \right\},
        \left\{ b, d \right\},
        \left\{ d, e \right\}
    \right\}
\end{equation}
Von (a) nach (b) wandert der Knoten $b$ in die Zusammenhangskomponente von $c$.
\begin{equation}
    \begin{split}
        Z_a' &= \left\{ a \right\} \\
        Z_c' &= \left\{ c, b, d \right\} \\
        Z_e' &= \left\{ e \right\}
    \end{split}
\end{equation}
Nun müssen nur noch $|E''| = 3$ Kanten entfernt werden.
Damit ist eine bessere Lösung gefunden.
\begin{equation}
    E'' = \left\{ 
        \left\{ a, b \right\},
        \left\{ a, e \right\},
        \left\{ d, e \right\}
    \right\}
\end{equation}


\begin{figure}[h]
    \centering
    \begin{subfigure}{0.45\textwidth}
        \centering
        \begin{tikzpicture}[
                scale=2,
            ]
            \tikzset{
                vertex/.style={
                    circle,
                    draw,
                    fill,
                }
            }
            \node[vertex, red!50!black,   label=left:$a$]  (a) at (0,0) {};
            \node[vertex, red!50!white,   label=left:$b$]  (b) at (0,1) {};
            \node[vertex, green!50!black, label=$c$]       (c) at (0,2) {};
            \node[vertex, green!50!white, label=$d$]       (d) at (1,2) {};
            \node[vertex, blue!50!black,  label=right:$e$] (e) at (2,1) {};
            \draw (a) to (b);
            \draw (a) to (e);
            \draw (b) to (c);
            \draw (b) to (d);
            \draw (c) to (d);
            \draw (d) to (e);
        \end{tikzpicture}
        \caption{vorher $|E'| = 4$}
        \label{subfig:pre}
    \end{subfigure}
    \begin{subfigure}{0.45\textwidth}
        \centering
        \begin{tikzpicture}[
                scale=2,
            ]
            \tikzset{
                vertex/.style={
                    circle,
                    draw,
                    fill,
                }
            }
            \node[vertex, red!50!black,   label=left:$a$]  (a) at (0,0) {};
            \node[vertex, green!50!white, label=left:$b$]  (b) at (0,1) {};
            \node[vertex, green!50!black, label=$c$]       (c) at (0,2) {};
            \node[vertex, green!50!white, label=$d$]       (d) at (1,2) {};
            \node[vertex, blue!50!black,  label=right:$e$] (e) at (2,1) {};
            \draw (a) to (b);
            \draw (a) to (e);
            \draw (b) to (c);
            \draw (b) to (d);
            \draw (c) to (d);
            \draw (d) to (e);
        \end{tikzpicture}
        \caption{nachher $|E'| = 3$}
        \label{subfig:post}
    \end{subfigure}
    \caption{Beispiel für eine bessere Lösung in der Nachbarschaft}
    \label{fig:local}
\end{figure}


\end{document}
