\documentclass[a4paper]{scrartcl}

% font/encoding packages
\usepackage[utf8]{inputenc}
\usepackage[T1]{fontenc}
\usepackage{lmodern}
\usepackage[ngerman]{babel}

% math
\usepackage{amsmath, amssymb, amsfonts, amsthm, mathtools}
\allowdisplaybreaks
\newtheorem*{behaupt}{Behauptung}
\usepackage{siunitx}

% misc
\usepackage{enumerate}
\usepackage{algorithm}
\usepackage{algpseudocode}
\usepackage[section]{placeins}

% tikz
\usepackage{tikz}
\usetikzlibrary{graphs}
\usetikzlibrary{arrows}
\usetikzlibrary{positioning}
\usepackage{pgfplots}
\usepgfplotslibrary{fillbetween}
                    
% macros
\newcommand{\linf}[1]{\lim_{#1 \to \infty}}
\newcommand{\gdw}{\Leftrightarrow}
\newcommand{\dif}{\mathop{}\!\mathrm{d}}
\newcommand{\bigo}{\mathcal{O}}


\title{Algorithmen und Datenstrukturen}
\subtitle{Übungsblatt 7 - Aufgabe 4}
\author{
    anonymous
}
\date{zum 27. Januar 2015}

\begin{document}
\maketitle

\begin{algorithm}
    \caption{\textsc{Set Cover: Branch and Bound}}
    \label{alg:bnb}
    \begin{algorithmic}[1]
        \Procedure{Set\_Cover\_BnB}{$A, S$}
        \State $S \gets \left\{ \{ S_i \} \text{ for } 1 \geq i \geq n \right\}$
        \State $bestsofar \gets \infty$
        \While {$S \neq \emptyset$}
            \State choose a $P \in S$
            \For {$S_k \in \left\{ S_i \ |\  i > \max\{ j \ |\  S_j \in P \} \right\}$}
                \State $P_k \gets P \cup S_k$
                \If {$\bigcup_{S_i \in P_k} S_i = A$}
                    \State $bestsofar \gets |P_k|$
                \ElsIf {$|P_k| < bestsofar$}
                \State $S \gets S \cup \{ P_k \}$
                \EndIf
            \EndFor
        \EndWhile
        \State \Return $bestsofar$
        \EndProcedure
    \end{algorithmic}
\end{algorithm}

Im Bound-Schritt wird die Mächtigkeit $|P_k|$ berechnet.
$P_k$ repräsentiert die Menge an Lösungen, welche die Teilmengen $S_i \in P_k$
verwenden, um $A$ zu überdecken.
Da in Zeile 10 $P_k \neq A$ gilt, müssen noch weitere Mengen hinzugenommen
werden.
Die Mächtigkeit von $P_k$ ist also eine untere Schranke der Anzahl an Mengen
für die betrachtete Lösungsmenge.

\end{document}
