\documentclass[a4paper]{scrartcl}

% font/encoding packages
\usepackage[utf8]{inputenc}
\usepackage[T1]{fontenc}
\usepackage{lmodern}
\usepackage[ngerman]{babel}

% math
\usepackage{amsmath, amssymb, amsfonts, amsthm, mathtools}
\allowdisplaybreaks
\newtheorem*{behaupt}{Behauptung}
\usepackage{siunitx}

% misc
\usepackage{enumerate}
\usepackage{algorithm}
\usepackage{algpseudocode}
\usepackage[section]{placeins}

% tikz
\usepackage{tikz}
\usetikzlibrary{graphs}
\usetikzlibrary{arrows}
\usetikzlibrary{positioning}
\usepackage{pgfplots}
\usepgfplotslibrary{fillbetween}
                    
% macros
\newcommand{\linf}[1]{\lim_{#1 \to \infty}}
\newcommand{\gdw}{\Leftrightarrow}
\newcommand{\dif}{\mathop{}\!\mathrm{d}}
\newcommand{\bigo}{\mathcal{O}}


\title{Algorithmen und Datenstrukturen}
\subtitle{Übungsblatt 3 - Aufgabe 5}
\author{
    anonymous
}
\date{zum 18. November 2014}

\begin{document}
\maketitle


\begin{algorithm}
    \caption{\textsc{Min-Max}}
    \begin{algorithmic}[1]
        \Procedure{Min-Max}{A}

        \If {$A[1] > A[2]$}
            \State $maxsofar \gets A[1]$
            \State $minsofar \gets A[2]$
        \Else
            \State $maxsofar \gets A[2]$
            \State $minsofar \gets A[1]$
        \EndIf

        \For {$i = 2$ \textbf{to} $\frac{n}{2}$}
            \If {$A[2i-1] < A[2i]$}
                \State $max \gets A[2i]$
                \State $min \gets A[2i-1]$
            \Else
                \State $max \gets A[2i-1]$
                \State $min \gets A[2i]$
            \EndIf
            \If {$max > maxsofar$}
                \State $maxsofar \gets max$
            \EndIf
            \If {$min < minsofar$}
                \State $minsofar \gets min$
            \EndIf
        \EndFor

        \State \Return $(minsofar, maxsofar)$

        \EndProcedure
    \end{algorithmic}
\end{algorithm}

Da wir annehmen, dass $n$ gerade ist, können wir sagen $n = 2k$ für ein
$k \geq 1$.
In der Schleife werden pro Iteration drei Vergleiche durchgeführt
(Zeilen 10, 17, 20).
Die Schleife wird $\frac{n}{2}-1$-mal durchlaufen ($i = 1$ wird ausgelassen).
Vor der Schleife wird ein Vergleich unabhängig von der Größe des Arrays
durchgeführt.
Insgesamt kommt man auf
\begin{equation}
    3 \cdot \left( \frac{n}{2} - 1 \right) + 1 = \frac{3n}{2} -2
    \text{ Vergleiche.}
\end{equation}


\end{document}
