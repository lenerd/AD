\documentclass[a4paper]{scrartcl}

% font/encoding packages
\usepackage[utf8]{inputenc}
\usepackage[T1]{fontenc}
\usepackage{lmodern}
\usepackage[ngerman]{babel}

% math
\usepackage{amsmath, amssymb, amsfonts, amsthm, mathtools}
\allowdisplaybreaks
\newtheorem*{behaupt}{Behauptung}
\usepackage{siunitx}

% misc
\usepackage{enumerate}
\usepackage{algorithm}
\usepackage{algpseudocode}
\usepackage[section]{placeins}

% tikz
\usepackage{tikz}
\usetikzlibrary{graphs}
\usetikzlibrary{arrows}
\usetikzlibrary{positioning}
\usepackage{pgfplots}
\usepgfplotslibrary{fillbetween}
                    
% macros
\newcommand{\linf}[1]{\lim_{#1 \to \infty}}
\newcommand{\gdw}{\Leftrightarrow}
\newcommand{\dif}{\mathop{}\!\mathrm{d}}
\newcommand{\bigo}{\mathcal{O}}


\title{Algorithmen und Datenstrukturen}
\subtitle{Übungsblatt 3 - Aufgabe 4}
\author{
    anonymous
}
\date{zum 18. November 2014}

\begin{document}
\maketitle


\begin{enumerate}[(a)]
    \item \hfill \\
        \begin{figure}[h]
            \centering
            \begin{tikzpicture}[
                    level 1/.style={sibling distance=16em},
                    level 2/.style={sibling distance=8em},
                    level 3/.style={sibling distance=4em},
                ]
                \node {$c < b$}
                    child
                    {
                        node {$b < a$}
                        child
                        {
                            node {$c < b$}
                            child
                            {
                                node {$[a,b,c]$}
                                edge from parent node [left] {$\bot$}
                            }
                            % child
                            % {
                            %     node {$[x,x,x]$}
                            %     edge from parent node [right] {$\top$}
                            % }
                            edge from parent node [left] {$\bot$}
                        }
                        child
                        {
                            node {$c < a$}
                            child
                            {
                                node {$[b,a,c]$}
                                edge from parent node [left] {$\bot$}
                            }
                            child
                            {
                                node {$[b,c,a]$}
                                edge from parent node [right] {$\top$}
                            }
                            edge from parent node [right] {$\top$}
                        }
                        edge from parent node [left] {$\bot$}
                    }
                    child
                    {
                        node {$c < a$}
                        child
                        {
                            node {$b < c$}
                            child
                            {
                                node {$[a,c,b]$}
                                edge from parent node [left] {$\bot$}
                            }
                            % child
                            % {
                            %     node {$[x,x,x]$}
                            %     edge from parent node [right] {$\top$}
                            % }
                            edge from parent node [left] {$\bot$}
                        }
                        child
                        {
                            node {$b < a$}
                            child
                            {
                                node {$[c,a,b]$}
                                edge from parent node [left] {$\bot$}
                            }
                            child
                            {
                                node {$[c,b,a]$}
                                edge from parent node [right] {$\top$}
                            }
                            edge from parent node [right] {$\top$}
                        }
                        edge from parent node [right] {$\top$}
                    };
            \end{tikzpicture}
            \caption{Berechnungsbaum des ursprünglichen Bubblesort}
        \end{figure}

    \item
        Wenn die innere Schleife über $j = L.length \textbf{ down to } i + 1$
        keine Elemente vertauscht, dann wird in der nächsten Iteration von der
        äußeren Schleife auch nichts mehr vertauscht (innere Schleife über
        $j = L.length \textbf{ down to } (i + 1) + 1$).
        Man kann in dem Fall den Algorithmus abbrechen, da die Liste sortiert
        ist.

    \item Siehe Algorithmus \ref{alg:bubble2} auf Seite \pageref{alg:bubble2}.
        \begin{algorithm}[h]
            \caption{\textsc{Bubblesort'}}
            \label{alg:bubble2}
            \begin{algorithmic}[1]
                \Procedure{Bubblesort'}{L}
                \For {$i = 1$ \textbf{to} $L.length$}
                    \State $swapped \gets false$
                    \For {$j = L.length$ \textbf{down to} $i + 1$}
                        \If {$L[j] < L[j - 1]$}
                            \State \Call{Swap}{$L[j], L[j - 1]$}
                            \State $swapped \gets true$
                        \EndIf
                    \EndFor
                    \If {$\lnot swapped$}
                        \State \Return
                    \EndIf
                \EndFor
                \EndProcedure
            \end{algorithmic}
        \end{algorithm}
        % # Python Implementierung
        % def bubble2(L):
        %     for i in range(1 - 1, len(L) - 1):
        %         swapped = False
        %         for j in range(len(L) - 1, (i + 1) - 1, -1):
        %             if L[j] < L[j - 1]:
        %                 t = L[j]
        %                 L[j] = L[j - 1]
        %                 L[j - 1] = t
        %                 swapped = True
        %         if not swapped:
        %             return L
        %     return L

    \item
        Der Worst Case besteht darin, dass die Liste zu Begin schon sortiert
        ist, jedoch in der falschen Reihenfolge (absteigend statt aufsteigend
        und anders herum).
        Der Mehraufwand besteht darin, dass bei jeder Vertauschung die
        Variable $swapped$ auf true gesetzt wird, sowie in der Überprüfung
        in Zeile 10.
        Somit werden mehr Anweisungen als im ursprünglichen Algorithmus
        ausgeführt.
        Die Laufzeit einer Implementation kann sich daher verlängern.
        In der $\mathcal{O}$-Notation ändert sich nichts an der Zeitkomplexität.

        Die Adaption lohnt sich dennoch, da der Worst Case sehr unwahrscheinlich
        ist und eine teilsortierte Liste wesentlich häufiger auftritt.
        Im Schnitt verbraucht der verbesserte Bubblesort weniger Zeit.

    \item
        Der Unterschied besteht darin, dass nach den zwei Vergleichen mit
        negativem Ergebnis (im Pfad ganz links) abgebrochen wird und das
        Ergebnis ein Schritt früher zur Verfügung steht.
        Es handelt sich um den Fall, dass die Liste $[a,b,c]$ bereits zu Beginn
        in der richtigen Reihenfolge ist.

        \begin{figure}[h]
            \centering
            \begin{tikzpicture}[
                    level 1/.style={sibling distance=16em},
                    level 2/.style={sibling distance=8em},
                    level 3/.style={sibling distance=4em},
                ]
                \node {$c < b$}
                    child
                    {
                        node {$b < a$}
                        child
                        {
                            node {$[a,b,c]$}
                            edge from parent node [left] {$\bot$}
                        }
                        child
                        {
                            node {$c < a$}
                            child
                            {
                                node {$[b,a,c]$}
                                edge from parent node [left] {$\bot$}
                            }
                            child
                            {
                                node {$[b,c,a]$}
                                edge from parent node [right] {$\top$}
                            }
                            edge from parent node [right] {$\top$}
                        }
                        edge from parent node [left] {$\bot$}
                    }
                    child
                    {
                        node {$c < a$}
                        child
                        {
                            node {$b < c$}
                            child
                            {
                                node {$[a,c,b]$}
                                edge from parent node [left] {$\bot$}
                            }
                            % child
                            % {
                            %     node {$[x,x,x]$}
                            %     edge from parent node [right] {$\top$}
                            % }
                            edge from parent node [left] {$\bot$}
                        }
                        child
                        {
                            node {$b < a$}
                            child
                            {
                                node {$[c,a,b]$}
                                edge from parent node [left] {$\bot$}
                            }
                            child
                            {
                                node {$[c,b,a]$}
                                edge from parent node [right] {$\top$}
                            }
                            edge from parent node [right] {$\top$}
                        }
                        edge from parent node [right] {$\top$}
                    };
            \end{tikzpicture}
            \caption{Berechnungsbaum des modifizierten Bubblesort}
        \end{figure}

\end{enumerate}


\end{document}
