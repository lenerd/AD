\documentclass[a4paper]{scrartcl}

% font/encoding packages
\usepackage[utf8]{inputenc}
\usepackage[T1]{fontenc}
\usepackage{lmodern}
\usepackage[ngerman]{babel}

% math
\usepackage{amsmath, amssymb, amsfonts, amsthm, mathtools}
\allowdisplaybreaks
\newtheorem*{behaupt}{Behauptung}
\usepackage{siunitx}

% tikz
\usepackage{tikz}
\usetikzlibrary{graphs}
\usetikzlibrary{arrows}
\usetikzlibrary{positioning}
\usepackage{pgfplots}
\usepgfplotslibrary{fillbetween}

% misc
\usepackage{enumerate}
\usepackage{algorithm}
\usepackage{algpseudocode}
\usepackage{listings}
\usepackage[section]{placeins}
                    
% macros
\newcommand{\linf}[1]{\lim_{#1 \to \infty}}
\newcommand{\gdw}{\Leftrightarrow}
\newcommand{\dif}{\mathop{}\!\mathrm{d}}
\newcommand{\bigo}{\mathcal{O}}

% last package
\usepackage{hyperref}


\title{Algorithmen und Datenstrukturen}
\subtitle{Übungsblatt 3 - Aufgabe 2}
\author{
    anonymous
}
\date{zum 18. November 2014}

\begin{document}
\maketitle


\begin{algorithm}
    \caption{\textsc{k-highest Comparison}}
    \begin{algorithmic}[1]
        \Procedure{Check-Node}{heap, index, x, k, count}
            \For{$i$ \textbf{in} $0, 1$}
                \If {$count \geq k$}
                    \State \Return $count$
                \EndIf
                \If {$heap[2 \cdot index + i] \geq x$}
                    \State $count \gets \Call{Check-Node}{heap, 2 \cdot index + i, k, x, count + 1}$
                \EndIf
            \EndFor
            \State \Return $count$
        \EndProcedure
        \Statex
        \Procedure{k-highest Comparison}{H, k, x}
            \If {$H[1] \geq x$}
                \State \Return $\Call{Check-Node}{H, 1, k, x, 1} \geq k$
            \EndIf
            \State \Return false
        \EndProcedure
    \end{algorithmic}
\end{algorithm}

Der Algorithmus schaut sich so lange rekursiv die Kinder ausgehend von der
Wurzel an, bis alle Elemente mit einem Wert $\geq x$ gefunden hat oder die
Anzahl dieser $\geq k$ wird.
Im ersten Fall, wird false zurückgegeben, in letzterem true.

\paragraph{Termination} \hfill \\
$count$ wird bei jedem rekursiven Aufruf inkrementiert.
Da $count \geq k$ die Abbruchbedingung der Rekursion ist, wird diese nach
endlich vielen Schritten beendet.

\paragraph{Korrektheit} \hfill \\
Anfang: \\
Wenn die Wurzel nicht $\geq x$ ist, dann ist es auch kein anderes Element des
Heap.
In diesem Fall wird direkt false zurückgegeben.
Andernfalls wird die Rekursion von \textsc{Check-Node} mit $count = 1$
gestartet.

Schritt: $index = i$, $count = n$ \\
Bisher wurden schon $n$ Einträge (inklusive des aktuellen Knotens) gefunden.
Gilt $n \geq k$, so wurden mindestens $k$ Elemente $\geq x$ gefunden.
Das $k$-größte Element ist also $\geq x$ und wir können die Rekursion abbrechen.
Im anderen Fall, wird \textsc{Check-Node} auf das linke Kind (Index $2i$)
angewandt.
Sind nun $n \geq k$, wird wie oben abgebrochen.
Sonst wird \textsc{Check-Node} auf das rechte Kind (Index $2i+1$) angewandt und
die Rückgabe zurückgegeben.

Sobald $k$ Elemente $\geq x$ gefunden wurden, wird also true zurückgegeben.
Werden alle Elemente $\geq x$ gefunden und deren Anzahl ist $< k$, so wird false
zurückgegeben.

\paragraph{Zeitkomplexität} \hfill \\
Da die Funktion \textsc{Check-Node} maximal $k$-mal aufgerufen wird (von $count$
beschränkt), liegt die Zeitkomplexität in $\mathcal{O}(k)$.

\paragraph{Platzkomplexität} \hfill \\
Da alle Informationen über den rekursiven Aufruf weitergegeben werden, liegen
die Daten auf dem Stack.
Da die maximale Rekursionstiefe $k$ beträgt, liegen zu jedem Zeitpunkt maximal
$k$ Callframes auf dem Stack.
Die Platzkomplexität liegt damit in $\mathcal{O}(k)$.

\end{document}
