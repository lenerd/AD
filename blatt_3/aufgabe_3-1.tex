\documentclass[a4paper]{scrartcl}

% font/encoding packages
\usepackage[utf8]{inputenc}
\usepackage[T1]{fontenc}
\usepackage{lmodern}
\usepackage[ngerman]{babel}

% math
\usepackage{amsmath, amssymb, amsfonts, amsthm, mathtools}
\allowdisplaybreaks
\newtheorem*{behaupt}{Behauptung}
\usepackage{siunitx}

% tikz
\usepackage{tikz}
\usetikzlibrary{graphs}
\usetikzlibrary{arrows}
\usetikzlibrary{positioning}
\usepackage{pgfplots}
\usepgfplotslibrary{fillbetween}

% misc
\usepackage{enumerate}
\usepackage{algorithm}
\usepackage{algpseudocode}
\usepackage{listings}
\usepackage[section]{placeins}
                    
% macros
\newcommand{\linf}[1]{\lim_{#1 \to \infty}}
\newcommand{\gdw}{\Leftrightarrow}
\newcommand{\dif}{\mathop{}\!\mathrm{d}}
\newcommand{\bigo}{\mathcal{O}}

% last package
\usepackage{hyperref}


\title{Algorithmen und Datenstrukturen}
\subtitle{Übungsblatt 3 - Aufgabe 1}
\author{
    anonymous
}
\date{zum 18. November 2014}

\begin{document}
\maketitle

Die Frage, ob sich in der einfach verketteten Liste eine Schleife befindet,
lässt sich mit dem Hase-Igel-Algorithmus lösen.

\begin{algorithm}
    \caption{\textsc{Hase-Igel}}
    \begin{algorithmic}[1]
        \Procedure{Hase-Igel}{L}
        \State $Hase \gets L.start.next$
        \State $Igel \gets L.start$
        \While {$Hase \neq NULL \land Hase \neq Igel$}
            \State $Hase \gets Hase.next$
            \State $Igel \gets Igel.next$
            \If {$Hase$}
                \State $Hase \gets Hase.next$
            \EndIf
        \EndWhile
        \If {$Hase$}
            \State \Return $\top$
        \Else
            \State \Return $\bot$
        \EndIf
        \EndProcedure
    \end{algorithmic}
\end{algorithm}

Der Hase-Igel-Algorithmus verwendet zwei zusätzliche Pointer auf Elemente der
Liste.
Damit liegt der Speicherverbrauch in $\mathcal{O}(1)$.

\begin{enumerate}
    \item Die Liste enthält keine Schleife. \\
        Da der Hase sich in jeder Iteration zwei Elemente weiter bewegt, trifft
        er nach $\frac{n}{2}$ Schritten auf den Nullpointer am Ende der Liste.
        Die Schleife wird abgebrochen und es wird $\bot$ zurückgegeben.
        Die Laufzeit liegt in diesem Fall in $\mathcal{O}(n)$.

    \item Die Liste enthält eine Schleife. \\
        Dann gibt es ein $k$-tes Element ($k \leq n$), auf das der Pointer des letzten
        Listenelements zeigt.
        Nach $k$ Iterationen befindet sich der Igel am Beginn der Schleife und
        bräuchte noch $n-k$ Schritte zum Ende der Schleife.
        In dieser Zeit bewegt sich der Hase durch die Schleife.
        Beide treffen sich frühstens nach $k$, spätestens nach $n$ Schritten.
        Die Schleife wird abgebrochen und es wird $\top$ zurückgegeben.
        Die Laufzeit liegt in diesem Fall ebenfalls in $\mathcal{O}(n)$.

\end{enumerate}


\end{document}
