\documentclass[a4paper]{scrartcl}

% font/encoding packages
\usepackage[utf8]{inputenc}
\usepackage[T1]{fontenc}
\usepackage{lmodern}
\usepackage[ngerman]{babel}

% math
\usepackage{amsmath, amssymb, amsfonts, amsthm, mathtools}
\allowdisplaybreaks
\newtheorem*{behaupt}{Behauptung}
\usepackage{siunitx}

% tikz
\usepackage{tikz}
\usetikzlibrary{graphs}
\usetikzlibrary{arrows}
\usetikzlibrary{positioning}
\usepackage{pgfplots}
\usepgfplotslibrary{fillbetween}

% misc
\usepackage{enumerate}
\usepackage{algorithm}
\usepackage{algpseudocode}
\usepackage{listings}
\usepackage[section]{placeins}
                    
% macros
\newcommand{\linf}[1]{\lim_{#1 \to \infty}}
\newcommand{\gdw}{\Leftrightarrow}
\newcommand{\dif}{\mathop{}\!\mathrm{d}}
\newcommand{\bigo}{\mathcal{O}}

% last package
\usepackage{hyperref}


\title{Algorithmen und Datenstrukturen}
\subtitle{Übungsblatt 3 - Aufgabe 3}
\author{
    anonymous
}
\date{zum 18. November 2014}

\begin{document}
\maketitle

Eine einfache Möglichkeit, einen vergleichsbasierten Algorithmus zu
modifizieren, ist die Folgende: \\
Für jedes Element wird zu Beginn des Verfahrens die ursprüngliche Position
gespeichert und bei Wertegleichheit als zweiter Schlüssel verwendet.

Man kann zum Beispiel die vorhandene Datenstruktur erweitern und die alte
Position als eine neue Eigenschaft hinzufügen.
Für ein reines Integerarray biete sich an, ein zweites Array mit den alten
Positionen von $1$ bis $n$ zu füllen.
Jedes mal, wenn zwei Elemente den Platz tauschen, wird diese swap-Operation
auch auf das neue Array mit den gleichen Indizes angewandt.

Die modifizierte Version kommt mit maximal der doppelten Anzahl an Vergleichen
aus, da bei jedem Vergleich im ursprünglichen Algorithmus maximal ein
zusätzlicher dazukommen kann.

\end{document}
