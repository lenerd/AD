\documentclass[a4paper]{scrartcl}

% font/encoding packages
\usepackage[utf8]{inputenc}
\usepackage[T1]{fontenc}
\usepackage{lmodern}
\usepackage[ngerman]{babel}

% math
\usepackage{amsmath, amssymb, amsfonts, amsthm, mathtools}
\allowdisplaybreaks
\newtheorem*{behaupt}{Behauptung}
\usepackage{siunitx}

% tikz
\usepackage{tikz}
\usetikzlibrary{graphs}
\usetikzlibrary{arrows}
\usetikzlibrary{positioning}
\usepackage{pgfplots}
\usepgfplotslibrary{fillbetween}

% misc
\usepackage{enumerate}
\usepackage{algorithm}
\usepackage{algpseudocode}
\usepackage{listings}
\usepackage[section]{placeins}
                    
% macros
\newcommand{\linf}[1]{\lim_{#1 \to \infty}}
\newcommand{\gdw}{\Leftrightarrow}
\newcommand{\dif}{\mathop{}\!\mathrm{d}}
\newcommand{\bigo}{\mathcal{O}}

% last package
\usepackage{hyperref}


\title{Algorithmen und Datenstrukturen}
\subtitle{Übungsblatt 4 - Aufgabe 2}
\author{
    anonymous
}
\date{zum 2. Dezember 2014}

\begin{document}
\maketitle

\begin{enumerate}[(a)]
    \item
        Die Aussage ist richtig. \\
        Sei $\pi$ ein Pfad durch $G$.
        Angenommen ein kürzester Pfad zwischen zwei Knoten $u,v \in V$ enthält
        einen Knoten $w \in V$ mehrfach.
        Dann hat der Pfad die Form $\pi = uxwywzv$ für Knotenfolgen $x, y, z$.
        Da $G$ keine Schleifen enthält, muss $y \neq \lambda$ gelten
        ($y$ darf nicht leer sein.).
        Offensichtlich ist der Pfad $\pi' = uxwzv$ kürzer als $\pi$.
        Dies ist ein Widerspruch zu der Annahme, dass der kürzeste Pfad einen
        Knoten mehrfach enthält.

    \item
        Die Aussage ist richtig. \\
        Da ein kürzester Pfad in $G$ immer ein einfacher Pfad ist (a), verläuft
        der kürzeste Pfad im längsten Fall einmal durch alle $n$ Knoten.
        Da auf einem Pfad mit $n$ Knoten $n-1$ Kanten liegen, gilt die Aussage.

    \item
        Die Aussage ist falsch. \\
        In Abbildung \ref{fig:tree-path} ist ein Baum $G'$ zu sehen.
        \begin{equation}
            G' = (V', E') = \left( \{a,b,c\}, \{ (a,b), (b,a), (b,c), (c,b) \} \right)
        \end{equation}
        Der rot markierte Pfad $\pi = abcba$ ist kein einfacher Pfad, da $a$
        und $b$ jeweils zweimal auftreten.
        \begin{figure}[h]
            \centering
            \begin{tikzpicture}[
                semithick,
                auto,
            ]
                \tikzstyle{path}=[->, >=stealth', red, bend left]
                \node [circle, draw] (a) [] {$a$};
                \node [circle, draw] (b) [right= of a] {$b$};
                \node [circle, draw] (c) [right= of b] {$c$};
                \draw (a) to (b);
                \draw (b) to (c);
                \draw [path] (a) to (b);
                \draw [path] (b) to (c);
                \draw [path] (c) to (b);
                \draw [path] (b) to (a);
            \end{tikzpicture}
            \caption{Pfad in einem Baum}
            \label{fig:tree-path}
        \end{figure}

    \item
        Die Aussage ist falsch. \\
        Ein Baum muss zusammenhängend sein.
        Beispielsweise betrachten wir folgenden Graphen
        \begin{equation}
            G' = (V', E') = (\{a, b\}, \emptyset)
        \end{equation}
        $G'$ besteht aus zwei Knoten ohne Kanten.
        Ohne Kanten können keine Zyklen vorhanden sein.
        Da $a$ und $b$ nicht verbunden sind, handelt es sich nicht um einen
        Baum, sondern um einen Wald mit zwei Bäumen.

    \item
        Die Aussage ist falsch. \\
        In Abbildung \ref{fig:3complete} besitzt jeder der drei Knoten den Grad
        $2$.
        Nach der Aussage müssten dann $6$ Kanten vorhanden sein.
        Dies ist offensichtlich nicht der Fall.
        Die Aussage müsste heißen: Die Hälfte der Summe aller Knotengrade in $G$
        ist gleich $m$.
        Jede Kante ist mit genau zwei Knoten verbunden, wird also in der Summe
        aller Grade doppelt gezählt.

    \item
        Die Aussage ist richtig. \\
        Der längste kürzeste Pfad zwischen zwei Knoten in einem $k$-nären Baum
        tritt auf, wenn sich beide Knoten auf der untersten Ebene befinden und
        der einzige gemeinsame Vorfahre die Wurzel des Baumes ist.
        Die Länge des Pfades ist also die doppelte Höhe des Baumes.

        Die Höhe des Baumes
        \begin{equation}
            h = \lceil \log_k (n+1) - 1 \rceil
        \end{equation}
        Für die Aussage müsste gelten:
        \begin{equation}
            \begin{split}
                diam(G) = 2h &\leq 2 \lceil \log_k n \rceil \\
                \gdw 2 \cdot \lceil \log_k (n+1) - 1 \rceil &\leq 2 \cdot \lceil \log_k n \rceil \\
                \gdw 2 \cdot \lceil \log_k (n+1) \rceil &\leq 2 \cdot \lceil \log_k n \rceil + 2 \\
                \gdw 2 \cdot \lceil \log_k (n+1) \rceil &\leq 2 \cdot \lceil \log_k n + 1 \rceil \\
                \gdw 2 \cdot \lceil \log_k (n+1) \rceil &\leq 2 \cdot \lceil \log_k (n \cdot k) \rceil \\
                \gdw n+1 &\leq n \cdot k
            \end{split}
        \end{equation}
        Für alle $k \geq 2$ und $n \geq 1$ gilt die Ungleichung.
        Daher gilt die Aussage.
        

    \item
        Die Aussage ist richtig. \\
        $G$ hat am meisten Kanten, wenn $G$ ein vollständiger Graph ist.
        In einem vollständigen Graphen sind alle Knoten paarweise miteinander
        verbunden.
        Von jedem der $n$ Knoten führt eine Kante zu jedem der anderen $n-1$
        Knoten.
        Nach der korrigierten Aussage in (e) muss die Anzahl der Kanten
        \begin{equation}
            m = \frac{n \cdot (n-1)}{2} \text{ sein.}
        \end{equation}
        

    \item
        Die Aussage ist falsch. \\
        Sei $G' = (V', E')$ ein vollständiger Graph mit $n \geq 3$.
        Zu jedem Knotenpaar $u,v \in V'$ gibt es  einen dritten Knoten
        $w \in V$, so dass der Pfad $uvw$ die Länge $2$ besitzt.
        Ein Beispiel ist in Abbildung \ref{fig:3complete} zu sehen.


\end{enumerate}
\begin{figure}[h]
    \centering
    \begin{tikzpicture}[
        semithick,
        auto,
    ]
        \node [circle, draw] (a) [] {};
        \node [circle, draw] (b) [below left= of a] {};
        \node [circle, draw] (c) [below right= of a] {};
        \draw (a) to (b);
        \draw (a) to (c);
        \draw (b) to (c);
    \end{tikzpicture}
    \caption{Vollständiger Graph mit $n = 3$}
    \label{fig:3complete}
\end{figure}

\end{document}
