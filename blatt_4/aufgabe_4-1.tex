\documentclass[a4paper]{scrartcl}

% font/encoding packages
\usepackage[utf8]{inputenc}
\usepackage[T1]{fontenc}
\usepackage{lmodern}
\usepackage[ngerman]{babel}

% math
\usepackage{amsmath, amssymb, amsfonts, amsthm, mathtools}
\allowdisplaybreaks
\newtheorem*{behaupt}{Behauptung}
\usepackage{siunitx}

% misc
\usepackage{enumerate}
\usepackage{algorithm}
\usepackage{algpseudocode}
\usepackage[section]{placeins}

% tikz
\usepackage{tikz}
\usetikzlibrary{graphs}
\usetikzlibrary{arrows}
\usetikzlibrary{positioning}
\usepackage{pgfplots}
\usepgfplotslibrary{fillbetween}
                    
% macros
\newcommand{\linf}[1]{\lim_{#1 \to \infty}}
\newcommand{\gdw}{\Leftrightarrow}
\newcommand{\dif}{\mathop{}\!\mathrm{d}}
\newcommand{\bigo}{\mathcal{O}}


\title{Algorithmen und Datenstrukturen}
\subtitle{Übungsblatt 4 - Aufgabe 1}
\author{
    anonymous
}
\date{zum 2. Dezember 2014}

\begin{document}
\maketitle

\begin{enumerate}[(a)]
    \item $l=2$
        \paragraph{Strategie}
        Sei $n = m^2$ für $n, m \in \mathbb{N}$ die Anzahl der Sprossen.
        Wir teilen die Sprossen in $m$ Intervalle zu je $m$ Sprossen.
        Mit dem ersten Glas, das uns zur Verfügung steht, prüfen wir, in
        welchem Intervall die gesuchte Sprosse liegt.
        Wir lassen es nacheinander von der $km$-ten Sprosse für
        $k = 1, \ldots, m$ fallen.
        Zerbricht es nicht, so kann das Glas von jeder Sprosse der Leiter
        fallen gelassen werden, ohne dass es zerbricht.
        Zerbricht es bei der $k$-ten Iteration, so muss sich die gesuchte
        Sprosse im Intervall $\left( (k-1)m, km \right)$ befinden.
        Das zweite Glas lassen wir beginnend mit der $(k-1)m$-ten Sprosse
        Das zweite Glas lassen wir nacheinander (beginnend mit der niedrigsten)
        von allen Sprossen in diesem Intervall fallen.
        Sobald es kaputt geht, wissen wir, dass die vorherige Sprosse der
        gesuchten Höhe entspricht.

        Jedes der zwei Gläser wird höchstens $m$-mal fallen gelassen.
        Die Anzahl liegt also in $\mathcal{O}(2\sqrt{n})$ und damit auch in
        $o(n)$.

        % \begin{algorithm}
        %     \caption{\textsc{A}}
        %     \begin{algorithmic}[1]
        %         \Procedure{A}{2, n}

        %         \State $m \gets \sqrt{n}$
        %         \State $g \gets 2$

        %         \For {$i = 1$ \textbf{to} $m$}
        %             \Comment{Berechne Intervall}
        %             \If {\textbf{not} \Call{TesteStufe}{$m \cdot i$}}
        %                 \State $g \gets g - 1$
        %                 \State \textbf{break}
        %             \EndIf
        %         \EndFor

        %         \If {$g = 2$}
        %             \State \Return Das Glas kann von jeder Sprosse der Leiter
        %                 fallen gelassen werden.
        %         \EndIf

        %         \State $stufe \gets i \cdot (m - 1) + 1$

        %         \While {\Call{TesteStufe}{stufe}}
        %             \State $stufe \gets stufe + 1$
        %             \Comment{Teste Stufen}
        %         \EndWhile

        %         \State $max \gets stufe - 1$

        %         \State \Return Das Glas kann höchstens von der $max$-ten Stufe
        %             fallen gelassen werden.

        %         \EndProcedure

        %         \Statex

        %         \Procedure{TesteStufe}{x}
        %             \State Glass von der $x$-ten Stufe fallen lassen
        %             \If {Glas noch heile}
        %                 \State \Return True
        %             \Else
        %                 \State \Return False
        %             \EndIf
        %         \EndProcedure
        %     \end{algorithmic}
        % \end{algorithm}

    \item

\end{enumerate}

\end{document}
