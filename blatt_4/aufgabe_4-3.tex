\documentclass[a4paper]{scrartcl}

% font/encoding packages
\usepackage[utf8]{inputenc}
\usepackage[T1]{fontenc}
\usepackage{lmodern}
\usepackage[ngerman]{babel}

% math
\usepackage{amsmath, amssymb, amsfonts, amsthm, mathtools}
\allowdisplaybreaks
\newtheorem*{behaupt}{Behauptung}
\usepackage{siunitx}

% tikz
\usepackage{tikz}
\usetikzlibrary{graphs}
\usetikzlibrary{arrows}
\usetikzlibrary{positioning}
\usepackage{pgfplots}
\usepgfplotslibrary{fillbetween}

% misc
\usepackage{enumerate}
\usepackage{algorithm}
\usepackage{algpseudocode}
\usepackage{listings}
\usepackage[section]{placeins}
                    
% macros
\newcommand{\linf}[1]{\lim_{#1 \to \infty}}
\newcommand{\gdw}{\Leftrightarrow}
\newcommand{\dif}{\mathop{}\!\mathrm{d}}
\newcommand{\bigo}{\mathcal{O}}

% last package
\usepackage{hyperref}


\title{Algorithmen und Datenstrukturen}
\subtitle{Übungsblatt 4 - Aufgabe 3}
\author{
    anonymous
}
\date{zum 2. Dezember 2014}

\begin{document}
\maketitle

\begin{enumerate}
    \item
        Sei $A \in \{0,1\}^{n \times n}$ die Adjazenzmatrix eines beliebigen
        schleifenfreien und ungewichteten Graphen $G = (V, E)$ und bezeichne
        $A[i,j]$ den Eintrag in der $i$-ten Zeile und $j$-ten Spalte.
        \begin{behaupt}
            Für alle $k \in \mathbb{N}$ ist $A^k[i,j]$ die Anzahl verschiedener
            Pfade der Länge $k$, die in $G$ von $i$ nach $j$ führen.
        \end{behaupt}
        \begin{proof}[Beweis durch vollständige Induktion] \hfill \\
            \textbf{Induktionsanfang} \\
            Sei $k = 0$.
            $A^0$ entspricht der Einheitsmatrix.
            \begin{equation}
                A^0 =
                \begin{pmatrix}
                    1 & 0 & \cdots & 0 \\
                    0 & 1 & \ddots & \vdots \\
                    \vdots & \ddots & \ddots & 0 \\
                    0 & \cdots & 0 & 1 \\
                \end{pmatrix}
            \end{equation}
            Besteht ein Pfad der Länge $0$ zwischen Knoten $i, j$, so gilt
            $i = j$.
            Pfade der Länge $0$ zwischen verschiedenen Knoten ohne
            Kantenübergänge sind nicht möglich.
            Es gibt nur die eine Möglichkeit, an einem Knoten zu bleiben, ohne
            über eine Kante zu gehen.
            Daher gilt die Behauptung für $k = 0$.
            
            Gelte die Behauptung für ein festes $k \in \mathbb{N}$.

            \textbf{Induktionsschritt} \\
            Seien $a,b,c \in V$.
            Gibt es einen Pfad der Länge $k$ von $i$ nach $l$ und einen Pfad der
            Länge $1$ von $l$ nach $j$, so existiert ein Pfad der Länge $k+1$
            zwischen $i$ und $j$.
            Die Anzahl der Pfade der Länge $k$ zwischen $i$ und $l$ ist nach
            Induktionsannahme $A^k[i,l]$.
            Es gilt für die Anzahl $m$ aller Pfade von $i$ über $l$ nach $j$
            \begin{equation}
                m = A^k[i,l] \cdot A[l,j] \text{ .}
                \label{eq:1}
            \end{equation}
            Um die Summe aller Pfade der Länge $k+1$ zwischen $i$ und $j$
            ($A^{k+1}[i,j]$) zu erhalten, muss $m$ für alle Knoten $l \in V$
            aufsummiert werden.
            \begin{equation}
                A^{k+1}[i,j] = \sum_{l=1}^n A^k[i,l] \cdot A[l,j]
            \end{equation}
            Dies entspricht der elementweisen Berechnung der
            Matrizenmultiplikation.
            \begin{equation}
                A^{k+1} = A^k \cdot A
            \end{equation}
            
            Damit ist die Behauptung durch vollständige Induktion gezeigt.
        \end{proof}

    \item
        Folgender Algorithmus berechnet eine Matrix $K$, mit welcher sich
        \emph{anschließend} in $\Theta(1)$ prüfen lässt, ob zwei Knoten
        $i,j \in V$ durch einen Pfad der Länge $k$ verbunden werden können.
        Ist dies der Fall, so gilt $K[i,j] \neq 0$, sonst ist $K[i,j] = 0$.
        \begin{algorithm}
            \caption{\textsc{k-Connection}}
            \begin{algorithmic}[1]
                \Procedure{k-Connection}{A, k}
                    \State \Return $A^k$
                \EndProcedure
            \end{algorithmic}
        \end{algorithm}

    \item
        Folgender Algorithmus berechnet die Potenzen $A^\kappa$ von $\kappa=0$
        bis maximal $\kappa=k$.
        Sobald in einer Iteration $A^\kappa[i,j] \neq 0$ ist, existiert ein
        Pfad der Länge höchstens $k$ von $i$ nach $j$; es wird True
        zurückgegeben.
        Existiert kein solcher Pfad, wird False zurückgegeben.
        \begin{algorithm}
            \caption{\textsc{Max-k-Connection}}
            \begin{algorithmic}[1]
                \Procedure{Max-k-Connection}{A, k, i, j}
                    \State $K \gets A^0$
                    \For {$\kappa = 1$ \textbf{to} $k$}
                        \If {$K[i,j] \neq 0$}
                            \State \Return True
                        \EndIf
                        \State $K \gets K \cdot A$
                    \EndFor
                    \State \Return $K[i,j] \neq 0$
                \EndProcedure
            \end{algorithmic}
        \end{algorithm}

\end{enumerate}

\end{document}
