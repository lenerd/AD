\documentclass[a4paper]{scrartcl}

% font/encoding packages
\usepackage[utf8]{inputenc}
\usepackage[T1]{fontenc}
\usepackage{lmodern}
\usepackage[ngerman]{babel}

% math
\usepackage{amsmath, amssymb, amsfonts, amsthm, mathtools}
\allowdisplaybreaks
\newtheorem*{behaupt}{Behauptung}
\usepackage{siunitx}

% misc
\usepackage{enumerate}
\usepackage{algorithm}
\usepackage{algpseudocode}
\usepackage[section]{placeins}

% tikz
\usepackage{tikz}
\usetikzlibrary{graphs}
\usetikzlibrary{arrows}
\usetikzlibrary{positioning}
\usepackage{pgfplots}
\usepgfplotslibrary{fillbetween}
                    
% macros
\newcommand{\linf}[1]{\lim_{#1 \to \infty}}
\newcommand{\gdw}{\Leftrightarrow}
\newcommand{\dif}{\mathop{}\!\mathrm{d}}
\newcommand{\bigo}{\mathcal{O}}


\title{Algorithmen und Datenstrukturen}
\subtitle{Übungsblatt 4 - Aufgabe 4}
\author{
    anonymous
}
\date{zum 2. Dezember 2014}

\begin{document}
\maketitle


\begin{behaupt}
    Es sind maximal $n-1$ Überweisungen notwendig, um die gegenseitigen
    Schulden von $n$ Personen auszugleichen.
\end{behaupt}
\begin{proof}
    Seien $s_i$ die Schulden (negativ) bzw. das Guthaben (positiv) der Person
    $i$.
    Da nur die Verhältnisse innerhalb dieser Menge an Personen betrachtet
    werden, muss gelten
    \begin{equation}
        \sum_{i=1}^n s_i = 0
    \end{equation}
    Person $n$ stellt ein Konto mit dem Betrag $k$ zur Verfügung.
    Der Saldo kann ohne Auswirkungen beliebig positiv oder negativ sein.
    Jede Person $i=1, \ldots, n-1$ hebt nun seinen Betrag $s_i$ von dem Konto
    ab (Für negative $s_i$, wird Geld auf das Konto überwiesen.).
    Nach diesen $n-1$ Transaktionen entspricht der Saldo auf dem Konto $k+s_n$.
    \begin{equation}
        k + \sum_{i=1}^n s_i - \sum_{i=1}^{n-1} s_i = k + s_n
    \end{equation}
    Da dieses schon der $n$-ten Person gehört, ist eine weitere Überweisung
    nicht notwendig.
\end{proof}

\end{document}
