\documentclass[a4paper]{scrartcl}

% font/encoding packages
\usepackage[utf8]{inputenc}
\usepackage[T1]{fontenc}
\usepackage{lmodern}
\usepackage[ngerman]{babel}

% math
\usepackage{amsmath, amssymb, amsfonts, amsthm, mathtools}
\allowdisplaybreaks
\newtheorem*{behaupt}{Behauptung}
\usepackage{siunitx}

% tikz
\usepackage{tikz}
\usetikzlibrary{graphs}
\usetikzlibrary{arrows}
\usetikzlibrary{positioning}
\usepackage{pgfplots}
\usepgfplotslibrary{fillbetween}

% misc
\usepackage{enumerate}
\usepackage{algorithm}
\usepackage{algpseudocode}
\usepackage{listings}
\usepackage[section]{placeins}
                    
% macros
\newcommand{\linf}[1]{\lim_{#1 \to \infty}}
\newcommand{\gdw}{\Leftrightarrow}
\newcommand{\dif}{\mathop{}\!\mathrm{d}}
\newcommand{\bigo}{\mathcal{O}}

% last package
\usepackage{hyperref}


\title{Algorithmen und Datenstrukturen}
\subtitle{Übungsblatt 1 - Aufgabe 1}
\author{
    anonymous
}
\date{zum 21. Oktober 2014}

\begin{document}
\maketitle

\begin{enumerate}[(a)]
    \item
        \begin{equation}
            \begin{gathered}
                \frac{1}{n}
                \prec
                1
                \prec
                \log \log n
                \prec
                \log n
                \asymp
                \log n^3
                \prec
                \log \left( n^{\log n} \right) \\
                \prec
                n^{\num{0,01}}
                \prec
                \sqrt{n}
                \prec
                n \log n
                \prec
                n^8
                \prec
                2^n
                \prec
                8^n
                \prec
                n!
                \prec
                n^n
            \end{gathered}
        \end{equation}
        Begründung
        \begin{enumerate}[i.]
            \item
                \begin{equation*}
                    \frac{1}{n} \prec 1
                \end{equation*}
                \begin{equation*}
                    \linf{n} \left( \frac{\frac{1}{n}}{1} \right)
                    = \linf{n} \left( \frac{1}{n} \right)
                    = 0 \Rightarrow \frac{1}{n} \in o \left( 1 \right)
                \end{equation*}
                
            \item
                \begin{equation*}
                    1 \prec \log \log n
                \end{equation*}

                \begin{equation*}
                    \linf{n} \left( \frac{1}{\log \log n} \right)
                    = 0 \Rightarrow 1 \in o \left( \log \log n \right)
                \end{equation*}

            \item
                \begin{equation*}
                    \log \log n \prec \log n
                \end{equation*}

                \begin{equation*}
                    \linf{n} \left( \frac{\log \log n}{\log n} \right)
                    = \linf{n} \left( \frac{\frac{1}{\log n} \cdot \frac{1}{n}}{\frac{1}{n}} \right)
                    = \linf{n} \left( \frac{1}{\log n} \right)
                    = 0 \Rightarrow \log \log n \in o \left( \log n \right)
                \end{equation*}

            \item
                \begin{equation*}
                    \log n \asymp \log n^3
                \end{equation*}

                \begin{equation*}
                    \begin{gathered}
                        \linf{n} \left( \frac{\log n}{\log n^3} \right)
                        = \linf{n} \left( \frac{\log n}{3 \cdot \log n} \right)
                        = \linf{n} \left( \frac{1}{3} \right)
                        = \frac{1}{3} \\
                        0 \leq \frac{1}{3} < \infty
                        \Rightarrow \log n \in \mathcal{O} \left( \log n^3 \right)
                        \\
                        \linf{n} \left( \frac{\log n}{\log n^3} \right)
                        = \linf{n} \left( \frac{\log n}{3 \cdot \log n} \right)
                        = \linf{n} \left( \frac{1}{3} \right)
                        = \frac{1}{3} \\
                        0 < \frac{1}{3} \leq \infty
                        \Rightarrow \log n \in \Omega \left( \log n^3 \right)
                        \\
                        \log n \in \mathcal{O} \left( \log n^3 \right)
                        \land
                        \log n \in \Omega \left( \log n^3 \right)
                        \Leftrightarrow
                        \log n \in \Theta \left( \log n^3 \right)
                    \end{gathered}
                \end{equation*}
            
            \item
                \begin{equation*}
                    \linf{n} \left( \frac{\log n^3}{\log \left( n^{\log n} \right)} \right)
                    = \linf{n} \left( \frac{3 \cdot \log n}{\log^2 n} \right)
                    = \linf{n} \left( \frac{3}{\log n} \right)
                    = 0 \Rightarrow \log n^3 \in o \left( \log \left( n^{\log n} \right) \right)
                \end{equation*}

            \item
                \begin{equation*}
                    \begin{gathered}
                        \linf{n} \left( \frac{\log \left( n^{\log n} \right)}{n^{\num{0,01}}} \right)
                        = \linf{n} \left( \frac{\log^2 n}{n^{\num{0,01}}} \right)
                        = \linf{n} \left( \frac{200 \cdot \log n}{n^{\num{0,01}}} \right)
                        = \linf{n} \left( \frac{20000}{n^{\num{0,01}}} \right)
                        = 0 \\
                        \Rightarrow \log \left( n^{\log n} \right) \in o \left( n^{\num{0,01}} \right)
                    \end{gathered}
                \end{equation*}

            \item
                \begin{equation*}
                    \linf{n} \left( \frac{n^{\num{0,01}}}{\sqrt{n}} \right)
                    = \linf{n} \left( n^{\num{-0,49}} \right)
                    = 0
                    \Rightarrow n^{\num{0,01}} \in o \left( \sqrt{n} \right)
                \end{equation*}

            \item
                \begin{equation*}
                    \linf{n} \left( \frac{\sqrt{n}}{n \cdot \log n} \right)
                    = \linf{n} \left( \frac{1}{\sqrt{n} \cdot \log n} \right)
                    = 0
                    \Rightarrow \sqrt{n} \in o \left( n \cdot \log n \right)
                \end{equation*}

            \item
                \begin{equation*}
                    \linf{n} \left( \frac{n \cdot \log n}{n^8} \right)
                    = \linf{n} \left( \frac{\log n + 1}{8n^7} \right)
                    = \linf{n} \left( \frac{1}{56n^7} \right)
                    = 0
                    \Rightarrow n \cdot \log n \in o \left( n^8 \right)
                \end{equation*}

            \item
                \begin{equation*}
                    \linf{n} \left( \frac{n^8}{2^n} \right)
                    = \linf{n} \left( \frac{8n^7}{\log 2 \cdot 2^n} \right)
                    = \cdots
                    = \linf{n} \left( \frac{8!}{\log^n 2 \cdot 2^n} \right)
                    = 0
                    \Rightarrow n^8 \in o \left( 2^n \right)
                \end{equation*}

            \item
                \begin{equation*}
                    \linf{n} \left( \frac{2^n}{8^n} \right)
                    = \linf{n} \left( \frac{1}{4}^n \right)
                    = 0
                    \Rightarrow 2^n \in o \left( 8^n \right)
                \end{equation*}

            \item
                \begin{equation*}
                    \begin{gathered}
                        \linf{n} \left( \frac{8^n}{n!} \right)
                        = \linf{n} \left( \underbrace{\frac{8}{1} \cdot \frac{8}{2} \cdot \cdots \cdot \frac{8}{8}}_{k} \cdot \frac{8}{9} \cdot \cdots \cdot \frac{8}{n-1} \cdot \frac{8}{n} \right)
                        \stackrel{(\star)}{=} 0 \\
                        \Rightarrow 8^n \in o \left( n! \right)
                    \end{gathered}
                \end{equation*}
                $(\star)$ Für alle $n \geq 8$ ist $k$ ein konstanter Faktor $> 1$.
                Strebt $n \to \infty$, kommen immer mehr Faktoren $< 1$ hinzu.
                

            \item
                \begin{equation*}
                    \begin{gathered}
                        \linf{n} \left( \frac{n!}{n^n} \right)
                        = \linf{n} \left( \frac{1}{n} \cdot \frac{2}{n} \cdot \cdots \cdot \frac{n-1}{n} \cdot \frac{n}{n} \right)
                            \stackrel{(\star)}{=} 0 \\
                            \Rightarrow n! \in o \left( n^n \right)
                        \end{gathered}
                \end{equation*}
                $(\star)$ Alle Faktoren sind $\leq 1$, deren Anzahl steigt mit
                $n$.

        \end{enumerate}

        \item
            \begin{enumerate}[(i)]
                \item
                    \begin{behaupt}
                        \begin{equation*}
                            \log_b n \in \Theta \left( \log_2 n \right) \text{ für } b > 1
                        \end{equation*}
                    \end{behaupt}
                    \begin{proof}
                        \begin{equation*}
                            \begin{gathered}
                                \linf{n} \left( \frac{\log_b n}{\log_2 n} \right)
                                = \linf{n} \left( \frac{\log_2 n}{\log_2 b \cdot \log_2 n} \right)
                                = \frac{1}{\log_2 b} \\
                                0 < \frac{1}{\log_2 b} < \infty
                                \Rightarrow \log_b n \in \Theta \left( \log_2 n \right)
                            \end{gathered}
                        \end{equation*}
                    \end{proof}

                \item
                    \begin{behaupt}
                        \begin{equation}
                            f \in \mathcal{O}(g) \not\Rightarrow g \in \omega(f)
                        \end{equation}
                    \end{behaupt}
                    \begin{proof}[Beweis durch Widerspruch] \hfill \\
                        Angenommen es gelte $f \in \mathcal{O}(g) \Rightarrow g \in \omega(f)$. \\
                        Sei $f \in \Theta(g)$.
                        Dann gelten $f \in \mathcal{O}(g)$ und $g \in \mathcal{O}(f)$.
                        Aus der Annahme folgt $g \in \omega(f)$, was ein
                        Widerspruch zu $g \in \mathcal{O}(f)$ ist
                        ($g$ kann nicht höchstens so schnell wie $f$ und schneller
                        als $f$ wachsen.).
                    \end{proof}

                \item
                    \begin{behaupt}
                        Für $f_c(n) = \sum_{i=0}^n c^i$ und $c \in \mathbb{R}^+$ gilt
                        \begin{equation}
                            f_c(n) \in \Theta(n) \gdw c = 1 \text{ .}
                        \end{equation}
                    \end{behaupt}
                    \begin{proof}
                        Die Behauptung kann umformuliert werden zu
                        \begin{gather}
                            c = 1 \Rightarrow f_c(n) \in \Theta(n)
                            \land
                            c \neq 1 \Rightarrow f_c(n) \not\in \Theta(n)
                            \text{ .}
                        \end{gather}
                        \begin{equation}
                            f_c(n) = \sum_{i=0}^{n}
                            =
                            \begin{cases}
                                n + 1 & \text{, wenn } c = 1 \\
                                \frac{1-c^{n+1}}{1-c} & \text{, wenn } c \neq 1 \\
                            \end{cases}
                        \end{equation}
                        Sei $c = 1$.
                        \begin{equation}
                            f_c(n) = n + 1 \in \Theta(n)
                        \end{equation}
                        Sei $c \neq 1$.
                        \begin{equation}
                            \begin{gathered}
                                \begin{split}
                                \linf{n} \left( \frac{\frac{1 - c^{n+1}}{1 - c}}{n} \right)
                                &= \frac{1}{1-c} \cdot \linf{n} \left( \frac{1-c^{n+1}}{n} \right) \\
                                &= \frac{1}{1-c} \cdot \linf{n} \left( -\log c \cdot c^{n+1} \right) \\
                                &= \underbrace{\frac{-\log c}{1-c}}_{> 0} \cdot \linf{n} \left( c^{n+1} \right) \\
                                &= \infty
                                \end{split} \\
                                \Rightarrow f_c(n) = \frac{1 - c^{n+1}}{1 - c} \not\in\Theta(n)
                            \end{gathered}
                        \end{equation}
                        
                        
                    \end{proof}

            \end{enumerate}
        
\end{enumerate}


\end{document}
