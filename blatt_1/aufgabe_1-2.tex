\documentclass[a4paper]{scrartcl}

% font/encoding packages
\usepackage[utf8]{inputenc}
\usepackage[T1]{fontenc}
\usepackage{lmodern}
\usepackage[ngerman]{babel}

% math
\usepackage{amsmath, amssymb, amsfonts, amsthm, mathtools}
\allowdisplaybreaks
\newtheorem*{behaupt}{Behauptung}
\usepackage{siunitx}

% tikz
\usepackage{tikz}
\usetikzlibrary{graphs}
\usetikzlibrary{arrows}
\usetikzlibrary{positioning}
\usepackage{pgfplots}
\usepgfplotslibrary{fillbetween}

% misc
\usepackage{enumerate}
\usepackage{algorithm}
\usepackage{algpseudocode}
\usepackage{listings}
\usepackage[section]{placeins}
                    
% macros
\newcommand{\linf}[1]{\lim_{#1 \to \infty}}
\newcommand{\gdw}{\Leftrightarrow}
\newcommand{\dif}{\mathop{}\!\mathrm{d}}
\newcommand{\bigo}{\mathcal{O}}

% last package
\usepackage{hyperref}


\title{Algorithmen und Datenstrukturen}
\subtitle{Übungsblatt 1 - Aufgabe 2}
\author{
    anonymous
}
\date{zum 21. Oktober 2014}

\begin{document}
\maketitle

\begin{enumerate}[(a)]
    \item
        \begin{behaupt}
            \begin{equation}
                F_n \geq 2^{\num{0,5}n} \text{ für alle } n \geq 6
            \end{equation}
        \end{behaupt}
        \begin{proof}[Beweis durch vollständige Induktion] \hfill \\
            \textbf{Induktions-Anfang} \\
            $n = 6$ \\
            \begin{equation}
                8 = F_6 \geq 2^{\num{0.5} \cdot 6} = 8
            \end{equation}
            $n = 7$ \\
            \begin{equation}
                13 = F_7 \geq 2^{\num{0.5} \cdot 7} \approx \num{11,31}
            \end{equation}
            Gelte die Behauptung für ein festes $n \in \mathbb{N}$.

            \textbf{Induktions-Schritt} \\
            \begin{equation}
                \begin{split}
                    F_{n+1} = F_{n} + F_{n-1} &\geq 2^{\num{0,5}n} + 2^{\num{0,5}(n-1)} \\
                    \gdw F_{n+1} &\geq 2^{\num{0,5}n} + 2^{\num{0,5}n} \cdot \frac{1}{\sqrt{2}} \\
                    \gdw F_{n+1} &\geq 2^{\num{0,5}n} \cdot \left( 1 + \frac{1}{\sqrt{2}} \right) \\
                    \gdw F_{n+1} &\geq 2^{\num{0,5}n} \cdot \left( 1 + \frac{1}{\sqrt{2}} \right)  \geq 2^{\num{0,5}n} \cdot \sqrt{2} \\
                    \gdw F_{n+1} &\geq 2^{\num{0,5}n} \cdot \left( 1 + \frac{1}{\sqrt{2}} \right)  \geq 2^{\num{0,5}(n+1)} \\
                    \gdw F_{n+1} &\geq  2^{\num{0,5}(n+1)}
                \end{split}
            \end{equation}
        \end{proof}

    \item
        \begin{behaupt}
            \begin{equation}
                F_n \leq 2^{cn} \text{ für alle } n \geq 0 \text{ und } c = \num{0,7}
            \end{equation}
        \end{behaupt}
        \begin{proof}[Beweis durch vollständige Induktion] \hfill \\
            \textbf{Induktions-Anfang} $n = 0$ \\
            \begin{equation}
                0 = F_0 \leq 2^{\num{0.7} \cdot 0} = 1
            \end{equation}
            Gelte die Behauptung für ein festes $n \in \mathbb{N}$.

            \textbf{Induktions-Schritt} \\
            \begin{equation}
                \begin{split}
                    F_{n+1} = F_n + F_{n-1} &\leq 2^{\num{0,7}n} + 2^{\num{0,7}(n-1)} \\
                    \gdw F_{n+1} &\leq 2^{\num{0,7}n} + 2^{\num{0,7}n} \cdot 2^{\num{-0,7}} \\
                    \gdw F_{n+1} &\leq 2^{\num{0,7}n} \cdot \left( 1 + 2^{\num{-0,7}} \right) \\
                    \gdw F_{n+1} &\leq 2^{\num{0,7}n} \cdot \left( 1 + 2^{\num{-0,7}} \right) \leq 2^{\num{0,7}n} \cdot 2^{\num{0,7}} \\
                    \gdw F_{n+1} &\leq 2^{\num{0,7}(n+1)} \\
                \end{split}
            \end{equation}
        \end{proof}
        Die kleinste obere Schranke ist
        $c = \frac{\log \phi}{\log 2} \approx \num{0.6942}$,
        wobei $\phi$ der Goldene Schnitt ist.
\end{enumerate}


\end{document}
