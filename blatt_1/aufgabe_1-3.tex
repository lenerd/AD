\documentclass[a4paper]{scrartcl}

% font/encoding packages
\usepackage[utf8]{inputenc}
\usepackage[T1]{fontenc}
\usepackage{lmodern}
\usepackage[ngerman]{babel}

% math
\usepackage{amsmath, amssymb, amsfonts, amsthm, mathtools}
\allowdisplaybreaks
\newtheorem*{behaupt}{Behauptung}
\usepackage{siunitx}

% misc
\usepackage{enumerate}
\usepackage{algorithm}
\usepackage{algpseudocode}
\usepackage[section]{placeins}

% tikz
\usepackage{tikz}
\usetikzlibrary{graphs}
\usetikzlibrary{arrows}
\usetikzlibrary{positioning}
\usepackage{pgfplots}
\usepgfplotslibrary{fillbetween}
                    
% macros
\newcommand{\linf}[1]{\lim_{#1 \to \infty}}
\newcommand{\gdw}{\Leftrightarrow}
\newcommand{\dif}{\mathop{}\!\mathrm{d}}
\newcommand{\bigo}{\mathcal{O}}


\title{Algorithmen und Datenstrukturen}
\subtitle{Übungsblatt 1 - Aufgabe 3}
\author{
    anonymous
}
\date{zum 21. Oktober 2014}

\begin{document}
\maketitle

\begin{enumerate}[(a)]
    \item
        \begin{behaupt}
            \begin{equation}
                \begin{pmatrix}
                    F_n \\ F_{n+1}
                \end{pmatrix}
                =
                \begin{pmatrix}
                    0 & 1 \\
                    1 & 1
                \end{pmatrix}^n
                \cdot
                \begin{pmatrix}
                    F_0 \\ F_1
                \end{pmatrix}
                \text{ für alle } n \geq 0
            \end{equation}
        \end{behaupt}
        \begin{proof}[Beweis durch vollständige Induktion] \hfill \\
            \textbf{Induktions-Anfang} $n = 0$ \\
            \begin{equation}
                \begin{pmatrix}
                    0 & 1 \\
                    1 & 1
                \end{pmatrix}^0
                \cdot
                \begin{pmatrix}
                    F_0 \\ F_1
                \end{pmatrix}
                =
                \begin{pmatrix}
                    1 & 0 \\
                    0 & 1
                \end{pmatrix}
                \cdot
                \begin{pmatrix}
                    F_0 \\ F_1
                \end{pmatrix}
                =
                \begin{pmatrix}
                    F_0 \\ F_1
                \end{pmatrix}
            \end{equation}
            Gelte die Behauptung für ein festes $n \in \mathbb{N}$.

            \textbf{Induktions-Schritt} \\
            \begin{equation}
                \begin{split}
                    \begin{pmatrix}
                        0 & 1 \\
                        1 & 1
                    \end{pmatrix}
                    \cdot
                    \begin{pmatrix}
                        F_n \\ F_{n+1}
                    \end{pmatrix}
                    &=
                    \begin{pmatrix}
                        0 \cdot F_0 + 1 \cdot F_{n+1} \\
                        1 \cdot F_0 + 1 \cdot F_{n+1} \\
                    \end{pmatrix} \\
                    \stackrel{(\star)}{\gdw}
                    \begin{pmatrix}
                        0 & 1 \\
                        1 & 1
                    \end{pmatrix}
                    \cdot
                    \begin{pmatrix}
                        F_n \\ F_{n+1}
                    \end{pmatrix}
                    &=
                    \begin{pmatrix}
                        F_n \\ F_{n+1}
                    \end{pmatrix} \\
                    \stackrel{(\star\star)}{\gdw}
                    \begin{pmatrix}
                        0 & 1 \\
                        1 & 1
                    \end{pmatrix}
                    \cdot
                    \begin{pmatrix}
                        0 & 1 \\
                        1 & 1
                    \end{pmatrix}^n
                    \cdot
                    \begin{pmatrix}
                        F_0 \\ F_1
                    \end{pmatrix}
                    &=
                    \begin{pmatrix}
                        F_n \\ F_{n+1}
                    \end{pmatrix} \\
                    \stackrel{(\star\star)}{\gdw}
                    \begin{pmatrix}
                        0 & 1 \\
                        1 & 1
                    \end{pmatrix}^{n+1}
                    \cdot
                    \begin{pmatrix}
                        F_0 \\ F_1
                    \end{pmatrix}
                    &=
                    \begin{pmatrix}
                        F_n \\ F_{n+1}
                    \end{pmatrix}
                \end{split}
            \end{equation}
            $(\star)$ Definition der Fibonacci-Folge \\
            $(\star\star)$ Induktions-Annahme
        \end{proof}

    \item
    \item
\end{enumerate}

\end{document}
