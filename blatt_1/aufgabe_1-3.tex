\documentclass[a4paper]{scrartcl}

% font/encoding packages
\usepackage[utf8]{inputenc}
\usepackage[T1]{fontenc}
\usepackage{lmodern}
\usepackage[ngerman]{babel}

% math
\usepackage{amsmath, amssymb, amsfonts, amsthm, mathtools}
\allowdisplaybreaks
\newtheorem*{behaupt}{Behauptung}
\usepackage{siunitx}

% misc
\usepackage{enumerate}
\usepackage{algorithm}
\usepackage{algpseudocode}
\usepackage[section]{placeins}

% tikz
\usepackage{tikz}
\usetikzlibrary{graphs}
\usetikzlibrary{arrows}
\usetikzlibrary{positioning}
\usepackage{pgfplots}
\usepgfplotslibrary{fillbetween}
                    
% macros
\newcommand{\linf}[1]{\lim_{#1 \to \infty}}
\newcommand{\gdw}{\Leftrightarrow}
\newcommand{\dif}{\mathop{}\!\mathrm{d}}
\newcommand{\bigo}{\mathcal{O}}


\title{Algorithmen und Datenstrukturen}
\subtitle{Übungsblatt 1 - Aufgabe 3}
\author{
    anonymous
}
\date{zum 21. Oktober 2014}

\begin{document}
\maketitle

\begin{enumerate}[(a)]
    \item
        \begin{behaupt}
            \begin{equation}
                \begin{pmatrix}
                    F_n \\ F_{n+1}
                \end{pmatrix}
                =
                \begin{pmatrix}
                    0 & 1 \\
                    1 & 1
                \end{pmatrix}^n
                \cdot
                \begin{pmatrix}
                    F_0 \\ F_1
                \end{pmatrix}
                \text{ für alle } n \geq 0
            \end{equation}
        \end{behaupt}
        \begin{proof}[Beweis durch vollständige Induktion] \hfill \\
            \textbf{Induktions-Anfang} $n = 0$ \\
            \begin{equation}
                \begin{pmatrix}
                    0 & 1 \\
                    1 & 1
                \end{pmatrix}^0
                \cdot
                \begin{pmatrix}
                    F_0 \\ F_1
                \end{pmatrix}
                =
                \begin{pmatrix}
                    1 & 0 \\
                    0 & 1
                \end{pmatrix}
                \cdot
                \begin{pmatrix}
                    F_0 \\ F_1
                \end{pmatrix}
                =
                \begin{pmatrix}
                    F_0 \\ F_1
                \end{pmatrix}
            \end{equation}
            Gelte die Behauptung für ein festes $n \in \mathbb{N}$.

            \textbf{Induktions-Schritt} \\
            \begin{equation}
                \begin{split}
                    \begin{pmatrix}
                        0 & 1 \\
                        1 & 1
                    \end{pmatrix}
                    \cdot
                    \begin{pmatrix}
                        F_n \\ F_{n+1}
                    \end{pmatrix}
                    &=
                    \begin{pmatrix}
                        0 \cdot F_n + 1 \cdot F_{n+1} \\
                        1 \cdot F_n + 1 \cdot F_{n+1} \\
                    \end{pmatrix} \\
                    \stackrel{(\star)}{\gdw}
                    \begin{pmatrix}
                        0 & 1 \\
                        1 & 1
                    \end{pmatrix}
                    \cdot
                    \begin{pmatrix}
                        F_n \\ F_{n+1}
                    \end{pmatrix}
                    &=
                    \begin{pmatrix}
                        F_n \\ F_{n+1}
                    \end{pmatrix} \\
                    \stackrel{(\star\star)}{\gdw}
                    \begin{pmatrix}
                        0 & 1 \\
                        1 & 1
                    \end{pmatrix}
                    \cdot
                    \begin{pmatrix}
                        0 & 1 \\
                        1 & 1
                    \end{pmatrix}^n
                    \cdot
                    \begin{pmatrix}
                        F_0 \\ F_1
                    \end{pmatrix}
                    &=
                    \begin{pmatrix}
                        F_n \\ F_{n+1}
                    \end{pmatrix} \\
                    \gdw
                    \begin{pmatrix}
                        0 & 1 \\
                        1 & 1
                    \end{pmatrix}^{n+1}
                    \cdot
                    \begin{pmatrix}
                        F_0 \\ F_1
                    \end{pmatrix}
                    &=
                    \begin{pmatrix}
                        F_n \\ F_{n+1}
                    \end{pmatrix}
                \end{split}
            \end{equation}
            $(\star)$ Definition der Fibonacci-Folge \\
            $(\star\star)$ Induktions-Annahme
        \end{proof}

    \item
        Mit der Methode \emph{exponentiation by squaring} lässt sich die Anzahl
        der Multiplikationen verringern.
        Handelt es sich bei dem Exponenten um eine Potenz von 2 ($n = 2^k$),
        so werden genau $\log n$ Multiplikationen benötigt.
        \begin{equation}
            X^{(2^k)} = \left( X^{(2^{k-1})} \right)^2
        \end{equation}

        Im allgemeinen Fall $n \in \mathbb{N}$ lautet die Formel wie folgt.
        \begin{equation}
            X^n = \begin{cases}
                \left( X^{\frac{n}{2}} \right)^2 & \text{ , wenn $n$ gerade} \\
                X \cdot \left( X^{\frac{n-1}{2}} \right)^2 & \text{ , wenn $n$ ungerade}
            \end{cases}
        \end{equation}

        Da das Problem in jedem Schritt halbiert wird, sind
        $\mathcal{O}\left( \log n \right)$ Schritte notwendig.

    \item
        Eine Matrixmultiplikation \textsc{MMul} setzt sich aus acht Multiplikationen
        ($\mathcal{O}\left( l^{\num{1,59}} \right)$) und vier Additionen
        ($\mathcal{O}(l)$) zusammen.
        Es gilt also $\textsc{MMul} \in \mathcal{O} \left( l^{\num{1,59}} \right)$.

        Die Potenz $\begin{pmatrix} 0 & 1 \\ 1 & 1 \end{pmatrix}^n$ kann mit
        $\mathcal{O} \left( \log n \right)$ Matrixmultiplikationen berechnet
        werden (s.\,o.).

        Für die Berechnung der Fibonacci-Zahlen mittels $2 \times 2$-Matritzen
        (\textsc{Fib\_Mat}) ist also ein Auwand von
        $\mathcal{O}\left( \log n \cdot l^{\num{1,59}} \right)$ notwendig.
        Dem gegenüber steht die iterative Berechnung aus der Vorlesung
        (\textsc{Fib\_Loop}) in $\mathcal{O} \left( n^2 \cdot l \right)$.
        Da die Bitlänge eines Integers sich während einer Berechnung sich
        normalerweise nicht ändert, handelt es sich bei $l$ um einen konstanten
        Wert.
        Aufgrund von
        \begin{equation}
            \linf{n} \left( \frac{\log n}{n^2} \right) = 0
        \end{equation}
        gilt
        \begin{equation}
            \textsc{Fib\_Mat} \in o \left( \textsc{Fib\_Loop} \right) \text{ .}
        \end{equation}
        Die Laufzeit von \textsc{Fib\_Mat} wächst also echt langsamer als die
        von \textsc{Fib\_Loop}.
        Damit ist \textsc{Fib\_Mat} echt schneller.

\end{enumerate}

\end{document}
